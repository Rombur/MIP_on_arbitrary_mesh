\section{Results} \label{sec_res}
In this section, we show two Fourier analysis: one where the $S_n$ order is
varied and one where the aspect ratio is varied. We also compare different
methods to solve MIP: congugate gradient (CG), conjugate gradient
preconditioned with symmetric Gauss-Seidel (PCG-SGS), conjugate gradient
preconditioned with ML using uncoupled aggregation (PCG-ML-Uncoupled),
conjugate gradient preconditioned with ML using mis aggregation (PCG-ML-MIS)
and AGMG.
\subsection{Fourier Analysis}
Fourier analysis is customary used to analyse Source Iteration accelerated
with DSA \cite{larsen_dsa,consistent_p1}. In a Fourier analysis, the error is
decomposed into modes that are characterized by a Fourier wave numbers. The
damping of these error modes during one step of Source Iteration + DSA
provides insight into the effectiveness of the acceleration method. The
largest damping factor over all wave numbers corresponds to the spectral
radius of the method. The smaller the spectral radius, the faster the
iterations will converge. If the spectral radius is greater than one, the
scheme is unstable. Next we present two Fourier analysis. In the first one,
the $S_n$ order is varied whereas in the second one, the aspect ratio of the 
cell is modified.
\subsubsection{$S_n$ order varied}
This Fourier analysis was carried on a square cell, using a
Gauss-Legendre-Chebyshev quadrature. The medium is homogeneous, the scattering
ratio  and periodic boundary conditions are used. The $x-$axis is the mesh
size in mean free path and the $y-$axis is the spectral radius. There are four
curves corresponding to different $S_n$ order: $S_2$, $S_4$, $S_8$ and
$S_{16}$ with $c=0.9999$.
\begin{figure}[H]
\centering
\includegraphics[width=0.5\textwidth]{sn_order_9999}
\caption{Fourier analysis as a function of the mesh optical thickness,
homogeneous infinite medium case.}
\end{figure}
MIP is stable for all the cell size. The spectral radius is always less than
0.5, except for $S_2$ where it is about 0.7.
\subsubsection{Aspect ratio varied}

\subsection{Homogeneous medium}
Next we compare different solvers for MIP on a homogeneous medium (10cm $\times$
10cm, $\Sigma_t=1$cm$^{-1}$ and $\Sigma_s=0.99$cm$^{-1}$), with vacuum boundary 
conditions and a source of intensity 1cm$^{-3}$s$^{-1}$. We use a $S_8$
Gauss-Legendre-Chebyshev quadrature, a Source Iteration solver with relative
tolerance of $10^{-8}$ and a relative tolerance for MIP of $10^{-10}$.
\subsubsection{Quadrilateral cells}
In this example, the mesh is composed of 9990 quadrilateral cells that corresponds 
to 39960 degrees of freedom. The solution is:
\begin{figure}[H]
\centering
\includegraphics[width=0.8\textwidth]{homog_quad_crop}
\caption{Quadrilateral cells.}
\end{figure}
In the next table, the different solvers are compared:
\begin{table}[H]
\begin{center}
\begin{tabular}{|c|c|c|c|c|c|c|}
\hline
 & No-DSA & CG & PCG-SGS & PCG-ML-Uncoupled & PCG-ML-MIS & AGMG\\
\hline
SI iterations & 268 & 27 & 27 & 27 & 27 & 27 \\
Precond init (s) & NA & NA & 0.0142572 & 0.381872 & 1.21095 & 0.068\\
MIP calculation (s) & NA & 138.379 & 177.556 & 46.1957 & 44.7837 & 6.9311\\
CG iterations & NA & 35419 & 11414 & 729 & 702 & 674\\
Total calculation (s) & 309.395 & 172.181 & 211.933 & 80.1548 & 80.0387 &
40.827\\
\hline
\end{tabular}
\caption{Comparison of preconditioners with quadrilateral cells.}
\end{center}
\end{table}
In this Table, SI iterations is the number iteration of Source Iteration
needed to solve the problem, Precond init is the time in seconds needed to
initialize the preconditioner used by CG, MIP calculation is the total time in
seconds spent solving DSA during the calculation, CG iterations is the total number 
of CG iterations used to solve MIP, and Total calculation is the time in
seconds needed to solve the problem.

Using MIP decreases significantly the number of SI iterations and the
calculation time. Using PCG-SGS decreases by a factor three the number of
CG iterations compared to CG but the time needed to solve MIP is higher. With
ML, the number of CG iterations is reduced by a factor 50 and the MIP
calculation time is divided by three. AGMG is by far the most efficient
solver, the number of CG iteration is slightly lower than PCG-ML but the MIP
calculation time is 20 times lower than CG.

\subsubsection{Polygonal cells}
In this problem the mesh is composed of 13654 triangles, 250 quadrilaterals,
1400 pentagons, 1306 hexagons, 228 heptagons, and 6 octagons, for a total of
16844 cells and 58442 degrees of freedoms. This example will allow us to test
MIP and the different preconditioners on a mesh composed of different types of
cell. The solution and the mesh are:
\begin{figure}[H]
\centering
\includegraphics[width=0.8\textwidth]{homog_poly_crop}
\caption{Polygonal cells.}
\end{figure}
The different solvers are compared in the next Table:
\begin{table}[H]
\begin{center}
\begin{tabular}{|c|c|c|c|c|c|c|}
\hline
 & No-DSA & CG & PCG-SGS & PCG-ML-Uncoupled & PCG-ML-MIS & AGMG\\
\hline
SI iterations & 268 & 27 & 27 & 27 & 27 & 27\\
Precond init (s) & NA & NA & 0.0201421 & 0.482845 & 2.09054 & 0.097\\
MIP calculation (s) & NA & 210.128 & 303.423 & 78.0993 & 71.4493 & 11.0419\\
CG iterations & NA & 36352 & 13370 & 840 & 784 & 674\\
Total calculation (s) & 458.132 & 265.026 & 358.745 & 134.434 & 128.731 &
66.0207\\
\hline
\end{tabular}
\caption{Comparison of preconditioners with polygonal cells.}
\end{center}
\end{table}
We see that using different types of cells in the same mesh does not affect
the performance of MIP or of the preconditioners. 

\subsection{Heterogeneous medium}
In this example, we use a heterogeneous medium composed of 64 triangles, 440
quadrilaterals and 331 hexagons for a total of 835 cells and 3938 degrees of
freedom. Reflective boundary conditions are used. The quadrature is a $S_{16}$
Gauss-Legendre-Chebyshev quadrature. The SI solver has a relative tolerance of 
$10^{-8}$ and the relative tolerance for MIP is $10^{-10}$. The domain is
composed of three zones:
\begin{figure}[H]
\centering
\includegraphics[width=0.4\textwidth]{source_crop}
\caption{Zones of the domain.}
\end{figure}
The properties of the different zones are:
\begin{description}
\item[Green zone:] $\Sigma_t =1$cm$^{-1}$, $\Sigma_s = 0.9$cm$^{-1}$, source$ =
1$cm$^{-3}$s$^{-1}$
\item[Red zone:] $\Sigma_t = 1.5$cm$^{-1}$, $\Sigma_s = 1.44$cm$^{-1}$, no source
\item[Blue zone:] $\Sigma_t = 1$cm$^{-1}$, $\Sigma_s = 0.3$cm$^{-1}$, no source
\end{description}
The scalar flux for this problem is:
\begin{figure}[H]
\centering
\includegraphics[width=0.6\textwidth]{heter_hexag_crop}
\caption{Hexagonal cells.}
\end{figure}
The different solvers are compared in the next Table:
\begin{table}[H]
\begin{center}
\begin{tabular}{|c|c|c|c|c|c|c|}
\hline
 & No-DSA & CG & PCG-SGS & PCG-ML-Uncoupled & PCG-ML-MIS & AGMG\\
\hline
SI iterations & 110 & 18 & 18 & 18 & 18 & 18\\
Precond init (s) & NA & NA &  0.00198889 & 0.049206 & 0.1407501 & 0.008\\
MIP calculation (s) & NA & 2.37668 & 5.98264 & 3.09196 & 3.00155 & 0.288559\\
CG iterations & NA & 4323 & 2747 & 355 & 341 & 245\\
Total calculation (s) & 30.7466 & 8.20233 & 11.5965 & 8.57338 & 8.67269 &
6.23169\\
\hline
\end{tabular}
\caption{Comparison of preconditioners with heterogeneous medium.}
\end{center}
\end{table}
We can see on this problem that MIP is effective even in heterogeneous
problem. However, because this problem is much smaller than the previous ones,
the preconditioner are not as effective than before. AGMG is the only MIP solver
faster than CG.
