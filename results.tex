%%%%%%%%%%%%%%%%%%%%%%%%%%%%%%%%%%%%%%%%%%%%%%%%%%%%%%%%%%%%%%%%%%%%%%%%%%%%%%%%%%
%%%%%%%%%%%%%%%%%%%%%%%%%%%%%%%%%%%%%%%%%%%%%%%%%%%%%%%%%%%%%%%%%%%%%%%%%%%%%%%%%%
\section{Results} \label{sec_res}
%%%%%%%%%%%%%%%%%%%%%%%%%%%%%%%%%%%%%%%%%%%%%%%%%%%%%%%%%%%%%%%%%%%%%%%%%%%%%%%%%%
%%%%%%%%%%%%%%%%%%%%%%%%%%%%%%%%%%%%%%%%%%%%%%%%%%%%%%%%%%%%%%%%%%%%%%%%%%%%%%%%%%

In this section, we present two series of results. First, Fourier analyses are 
carried out to analyze the performance of the MIP-DSA acceleration scheme 
for an homogeneous infinite medium meshed with rectangular cells and discretized with PWLD.
The effects of the $S_n$ order and the cell aspect ratio on the spectral radius of the iterative 
scheme are also studied. Second, the MIP-DSA technique is implemented in a 2D \sn code that uses
arbitrary polygonal grids with a PWLD spatial discretization.  Numerical examples employ
several mesh types: arbitrary quadrilaterals, arbitrary polygons, a regular layout of 
hexagons/triangles/rectangles, and a grid that mimics adaptive mesh refinement 
performed on rectangular cells. The MIP-DSA diffusion solves are performed using various
linear solvers: Conjugate Gradient (CG), Conjugate Gradient
Preconditioned with Symmetric Gauss-Seidel (PCG-SGS), Conjugate Gradient
Preconditioned with ML using Uncoupled aggregation (PCG-ML/U),
Conjugate Gradient Preconditioned with ML using MIS aggregation (PCG-ML/MIS),
and Conjugate Gradient Preconditioned with AGMG (AGMG). 

%%%%%%%%%%%%%%%%%%%%%%%%%%%%%%%%%%%%%%%%%%%%%%%%%%%%%%%%%%%%%%%%%%%%%%%%%%%%%%%%%%
\subsection{Fourier Analyses}
%%%%%%%%%%%%%%%%%%%%%%%%%%%%%%%%%%%%%%%%%%%%%%%%%%%%%%%%%%%%%%%%%%%%%%%%%%%%%%%%%%

Fourier analyses are often performed to assess some of the properties of 
DSA-accelerated transport solves \cite{dsa_ref,larsen_dsa,consistent_p1}. In a Fourier analysis,
the eigenvalues of the iteration matrix are analyzed, assuming a Fourier ansatz for the 
error modes. Specifically, the iteration matrices for the SI and SI+DSA schemes are given by
\begin{equation}
\bs{D L^{-1}M \Sigma} \quad \text{and} \quad \bs{I}-\bs{(I+A^{-1}\Sigma)(I-D L^{-1}M \Sigma)},
\end{equation}
respectively.  The error modes are of the form $\exp(i \bs{\Lambda} \cdot \bs{r})$, with the
wave number $\bs{\Lambda}=[\lambda_x,\, \lambda_y]^T$. This expression for the error modes 
is inserted into the discretized equations and the spectral radius (largest eigenvalue in magnitude)
of the iteration matrices are sought for $0 \le \lambda_x \le 2\pi/X$ and $0 \le \lambda_y \le 2\pi/Y$,
where $X$ and $Y$ are the dimensions of the rectangular domain.


\subsubsection{Spectral radius as a function of the $S_n$ order}
%%%%%%%%%%%%%%%%%%%%%%%%%%%%%%%%%%%%%%%%%%%%%%%%%%%%%%%%%%%%%%%%%%%%%%%%%%%%%%%%%%

This Fourier analysis is carried on a square cell ($X=Y$), using a
Gauss-Legendre-Chebyshev (GLC) angular quadrature. The medium is homogeneous with a 
scattering ratio $c=0.9999$; periodic boundary conditions are used. The results 
are plotted on \Cref{fig_fa_sn}, where the $x-$axis is the mesh size in mean free 
paths and the $y-$axis is the spectral radius. There are four curves corresponding 
to different $S_n$ orders: $S_2$, $S_4$, $S_8$ and $S_{16}$.
From \Cref{fig_fa_sn}, we conclude that MIP is stable for every 
cell size. The spectral radius is always less than 0.5, except for $S_2$ where 
it is about 0.7. In the fine mesh limit, the spatial continuum results are recovered \cite{multisweep}:
the spectral radius of SI+DSA using an $S_2$ quadrature in 2D undiscretized space is $0.5c$; as the angular 
quadrature is refined, the standard limit value of $0.2247c$ for the spectral result is obtained.
\begin{figure}[!htbp]
  \centering
  \includegraphics[width=0.7\textwidth]{sn_order_9999}
  \caption{Fourier analysis as a function of the mesh optical thickness, square cell,
    various \sn orders.}
  \label{fig_fa_sn}
\end{figure}

\subsubsection{Spectral radius as a function of the cell aspect ratio}
%%%%%%%%%%%%%%%%%%%%%%%%%%%%%%%%%%%%%%%%%%%%%%%%%%%%%%%%%%%%%%%%%%%%%%%%%%%%%%%%%%
For this Fourier analysis, we use a $S_{16}$ GLC quadrature. The medium is
again homogeneous with $c=0.9999$ and periodic boundary conditions apply. 
On \Cref{fig_fa_ar}, the five curves correspond to the following cell aspect 
ratios: $\frac{Y}{X}=\frac{1}{16}$, $\frac{1}{4}$,
$1$, $4$, $16$, and $100$.
We note that the MIP-DSA scheme is stable for every aspect ratio tested, including an
aspect ratio value of 100, 
and that the maximum spectral radius shows little sensitivity to the aspect ratio.
\begin{figure}[!htbp]
  \centering
  \includegraphics[width=0.7\textwidth]{aspect_ratio_9999_2}
  \caption{Fourier analysis as a function of the mesh optical thickness,
  $S_{16}$ angular quadrature, various aspect ratios.}
  \label{fig_fa_ar}
\end{figure}

%%%%%%%%%%%%%%%%%%%%%%%%%%%%%%%%%%%%%%%%%%%%%%%%%%%%%%%%%%%%%%%%%%%%%%%%%%%%%%%%%%
\subsection{Performance of MIP-DSA Implemented in a PWLD $S_n$ Transport Code}
%%%%%%%%%%%%%%%%%%%%%%%%%%%%%%%%%%%%%%%%%%%%%%%%%%%%%%%%%%%%%%%%%%%%%%%%%%%%%%%%%%
The MIP-DSA scheme has been implemented in a 2D \sn code that employs a PWLD discretization
for arbitrary polygonal grids. Several test cases are presented.

\subsubsection{Homogeneous test problems}  \label{sec_homog}
%%%%%%%%%%%%%%%%%%%%%%%%%%%%%%%%%%%%%%%%%%%%%%%%%%%%%%%%%%%%%%%%%%%%%%%%%%%%%%%%%%

We compare the different linear solvers employed for MIP-DSA using a homogeneous medium, $100cm
\times 100cm$, $\Sigma_t = 1cm^{-1}$ and $\Sigma_s = 0.999cm^{-1}$, with
vacuum boundary conditions and a unit source of intensity $1cm^{-3}s^{-1}$. We
use an $S_8$ GLC angular quadrature, Source Iteration as solver
with relative tolerance of $10^{-8}$ and a relative tolerance of
$10^{-10}$ for MIP-DSA solver. The medium is discretized using two different meshes:
\begin{enumerate}
  \item a quadrilateral grid composed of 49263 quadrilateral
    cells (197,052 degrees of freedom); 
  \item a polygonal grid composed of 45,204 triangles, 823
    quadrilaterals, 4,978 pentagons, 4,155 hexagons, 725 heptagons, and 24
    octagons, for a total of 55,909 cells and 193,991 degrees of freedom. This
    example will allow us to test MIP and the different preconditioners on a
    mesh composed of different cell types.
\end{enumerate}
%
The meshes and the numerical solutions are given on \Cref{homog_test_quads,homog_test_polys}.
In \Cref{comparison_homog_quad}, results obtained with the different linear solvers for MIP-DSA 
are compared for the quadrilateral grid.
In \Cref{comparison_homog_quad}, SI iterations is the number iterations of 
needed to solve the problem, Precond init is the time, in
seconds, needed to initialize the preconditioner used by CG, MIP calculation
is the total time, in seconds, spent solving DSA during the calculation, CG
iterations is the total number of CG iterations used to solve MIP, and Total
calculation is the time, in seconds, needed to solve the problem. 
We note that the accelerated transport solves only require 24 SI iterations, regardless of the linear solver
employed in MIP-DSA (as expected). Preconditioning CG significantly reduces the number of CG iterations.
%
We observe that algebraic multigrid processes, PGC-ML and AGMG, require 
about the same number of iterations (two orders of magnitude less than unpreconditioned CG). 
However, AMG is significantly faster than PCG-ML. We also note that PCG-SGS iteration is 
slower than one unpreconditioned CG iteration. Profiling of the code reveals that the
bottleneck is the function \emph{Ifpack\_PointRelaxation::ApplyInverseSGS\_FastCrsMatrix} 
of Trilinos. This function applies the forward and the backward substitutions required by SGS.
It is unclear why these substitutions are so costly. Also note that SGS is employed as
a pre- and post-smoother in the ML package of Trilinos and the same function
is once again the bottleneck of the method.
%
The different linear solvers are compared for the polygonal grid in \Cref{comparison_homog_poly}:
We note that using different spatial cell types in the same grid does not affect
the performance of MIP-DSA or that of its preconditioners.

\begin{figure}[!htbp]
  \centering
  \begin{subfigure}{0.45\textwidth}
    \centering
    \includegraphics[width=\textwidth]{quad_solu0000}
    \caption{Whole domain}
  \end{subfigure}
  \begin{subfigure}{0.45\textwidth}
    \centering
    \includegraphics[width=\textwidth]{quad_solu0001}
    \caption{Zoom}
  \end{subfigure}
  \caption{Grids and scalar flux solutions for the homogenous test problem using quadrilaterla cells}
  \label{homog_test_quads}
\end{figure}

\begin{figure}[!htbp]
  \centering
  \begin{subfigure}{0.45\textwidth}
    \centering
    \includegraphics[width=\textwidth]{polygon_solu0000}
    \caption{Whole domain}
  \end{subfigure}
  \begin{subfigure}{0.45\textwidth}
    \centering
    \includegraphics[width=\textwidth]{polygon_solu0001}
    \caption{Zoom}
  \end{subfigure}
  \caption{Grids and scalar flux solutions for the homogenous test problem  using arbitrary polygonal cells}
  \label{homog_test_polys}
\end{figure}

%
%\begin{figure}[!htbp]
  %\centering
  %\begin{subfigure}{0.75\textwidth}
    %\centering
    %\includegraphics[width=\textwidth]{big_homog_quad_crop}
    %\caption{Quadrilateral cells}
  %\end{subfigure}
  %\begin{subfigure}{0.75\textwidth}
    %\centering
    %\includegraphics[width=\textwidth]{big_homog_poly_crop}
    %\caption{Polygonal cells}
  %\end{subfigure}
  %\caption{Grids and scalar flux solutions for the homogenous material test problem}
  %\label{homog_test}
%\end{figure}
%
\begin{table}[!htbp]
  \begin{center}
    \caption{Comparison of different preconditioners for quadrilateral cells}
    \begin{tabular}{|c|c|c|c|c|c|c|}
      \hline
      & No-DSA & CG & PCG-SGS & PCG-ML/U & PCG-ML/MIS & AGMG \\
      \hline
      SI iterations   & 7311    & 24      & 24       & 24      & 24      & 24 \\
   Precond init (s)   & NA      & NA      & 0.171358 & 1.8255  & 9.56078 & 0.332 \\
MIP calculation (s)   & NA      & 1095.7  & 1311.76  & 192.622 & 197.632 & 29.9727 \\
      CG iterations   & NA      & 56649   & 17332    & 630     & 604     & 578 \\
Total calculation (s) & 39176.7 & 1264.98 & 1477.95  & 363.202 & 367.841 &
      194.568 \\
      \hline
    \end{tabular}
    \label{comparison_homog_quad}
  \end{center}
\end{table}
%
\begin{table}[!htbp]
  \begin{center}
    \caption{Comparison of different preconditioners for polygonal cells}
    \begin{tabular}{|c|c|c|c|c|c|c|}
      \hline
      & No-DSA & CG & PCG-SGS & PCG-ML/U & PCG-ML/MIS & AGMG \\
      \hline
      SI iterations   & 7311    & 23      & 23      & 23      & 23      & 23 \\
   Precond init (s)   & NA      & NA      & 0.06388 & 1.73379 & 8.0426  & 0.388 \\
MIP calculation (s)   & NA      & 877.861 & 1263.31 & 198.63  & 191.989 &
      31.242 \\
      CG iterations   & NA      & 46262   & 16712   & 652     & 603     & 555 \\
Total calculation (s) & 42666.7 & 1060.53 & 1447.53 & 382.275 & 384.422 &
      216.946 \\
      \hline
    \end{tabular}
    \label{comparison_homog_poly}
  \end{center}
\end{table}

\subsubsection{Heterogeneous medium}
%%%%%%%%%%%%%%%%%%%%%%%%%%%%%%%%%%%%%%%%%%%%%%%%%%%%%%%%%%%%%%%%%%%%%%%%%%%%%%%%%%

In this example, a heterogeneous geometry with three materials is used. It is 
composed of 184 triangles, 3,720 quadrilaterals and 2,791 regular hexagons of 
side $0.05cm$ for a total of 6,695 cells and 32,178 degrees of freedom (spatial 
unknowns per angular direction). The domain is $5.28275cm$ by $4.6cm$. 
Reflective boundary conditions are used. A $S_{16}$ GLC 
quadrature is used. The SI solver has a relative tolerance of 
$10^{-8}$ and the relative tolerance for MIP is $10^{-10}$. \Cref{hex_zones}
shows the problem geometry and the material properties are in given
\Cref{hex_prop}.
The different linear solvers for MIP-DSA are compared in \Cref{comparison_hex}.
%
\begin{figure}[!htbp]
  \centering
  \begin{subfigure}{0.45\textwidth}
    \centering
    \includegraphics[width=\textwidth]{hexa_grid0000}
    \caption{Whole domain}
  \end{subfigure}
  \begin{subfigure}{0.45\textwidth}
    \centering
    \includegraphics[width=\textwidth]{hexa_grid0001}
    \caption{Zoom}
  \end{subfigure}
  \caption{Material zones for the heterogeneous test problem}
  \label{hex_zones}
\end{figure}
%
\begin{table}[!htbp]
  \begin{center}
    \caption{MAterial Properties For the Different Regions}
    \begin{tabular}{|c|c|c|c|}
      \hline
       & Inner region & Intermediate region & Outer region \\ \hline
      $\Sigma_t$ $(cm^{-1})$ & 1.5 & 1.0 & 1.0 \\
      $\Sigma_s$ $(cm^{-1})$ & 1.4999 & 0.999 & 0.3 \\
     source $(cm^{-3}s^{-1}$ & 1.0 & 0.0 & 0.0 \\
      \hline
    \end{tabular}
    \label{hex_prop}
  \end{center}
\end{table}
%
%
\begin{table}[!htbp]
  \begin{center}
    \caption{Comparison of preconditioners (heterogeneous problem)}
    \begin{tabular}{|c|c|c|c|c|c|c|}
      \hline
      & No-DSA & CG & PCG-SGS & PCG-ML/U & PCG-ML/MIS & AGMG\\
      \hline
      SI iterations & 278     & 17      & 17        & 17       & 17      & 17  \\
   Precond init (s) & NA      & NA      & 0.0160661 & 0.368768 & 1.41632 &
      0.07  \\
MIP calculation (s) & NA      & 58.422  & 126.93    & 33.2225  & 31.3045 &
      2.924 \\
      CG iterations & NA      & 12214   & 6679      & 415      & 386     & 248  \\
Total calculation (s) & 910.566 & 120.889 & 190.413 & 99.7524  & 97.4666 &
      70.6424 \\      
      \hline
    \end{tabular}
    \label{comparison_hex}
  \end{center}
\end{table}
%
The remarks made in \Cref{sec_homog} for the homogeneous test problem
remain mostly unchanged. MIP-DSA is effective for this heterogeneous test case and AGMG is
still the fastest solver. It is interesting to note that, contrary to the
homogeneous tests where the number of CG iterations remained similar for all
algebraic multigrid preconditioners, in this test, AGMG requires
significantly fewer iterations than the Trilinos implementations, PCG-ML/U and and PCG-ML/MIS.

\subsubsection{Locally refined grid}
%%%%%%%%%%%%%%%%%%%%%%%%%%%%%%%%%%%%%%%%%%%%%%%%%%%%%%%%%%%%%%%%%%%%%%%%%%%%%%%%%%

In this example, a 3-material domain of size $10cm\times 10cm$ is used. 
\Cref{mat_amr} shows the material zoning and the mesh used. Material properties 
are given in \Cref{prop_amr}. The grid used mimics meshes obtained via adaptive mesh
refinement: the rectangular cells at the interfaces between two materials are refined once more,
leading to a grid composed of 10,482 quadrilaterals, 236 pentagons,
and 2 hexagons for a total of 10,720 cells and 43120 degrees of freedom. The bottom
and left sides of the domain have reflective boundary conditions, while the other two sides
have vacuum boundary conditions. 
%
The distribution of cells is given \Cref{fig_pol_dist}.
A $S_{16}$ GLC quadrature is employed. The tolerance on SI is $10^{-8}$ and
the tolerance on the CG solvers is $10^{-10}$.
The different linear solvers for MIP-DSA are compared in \Cref{table_amr}.
%
The conclusions drawn from this test case are similar to the ones made for 
our previous tests. This test case demonstrates that degenerate polygons 
(here, pentagons and hexagons) do not seem to affect the MIP-DSA acceleration.


\begin{figure}[!htbp]
  \centering
  \begin{subfigure}{0.45\textwidth}
    \centering
    \includegraphics[width=5cm]{zone_amr}
    \caption{Material regions}
    \label{mat_amr}
  \end{subfigure}
  \begin{subfigure}{0.45\textwidth}
    \centering
    \includegraphics[width=5cm]{amr_grid0001}
    \caption{Polygons distribution (zoom): quadrilaterals (blue cells),  pentagons(green cells),  hexagons (red cells)}
    \label{fig_pol_dist}
  \end{subfigure}
  \caption{AMR-like test domain}
\end{figure}
%
\begin{table}
  \begin{center}
    \caption{Material properties, AMR-like test problem}
    \begin{tabular}{|c|c|c|c|}
      \hline
      & Inner region & Intermediate region & Outer region  \\ \hline
    $\Sigma_t$ $(cm^{-1})$ & 1.5  & 1.0 & 1.0 \\
    $\Sigma_s$ $(cm^{-1})$ & 1.44 & 0.9 & 0.3 \\
  Source $(cm^{-3}s^{-1})$ & 1.0  & 0.0 & 0.0 \\
      \hline
    \end{tabular}
    \label{prop_amr}
  \end{center}
\end{table}

where the blue cells are quadrilaterals, the green cells are pentagons, and
the red cells are hexagons. This mesh is typical of a mesh obtained after one 
level of adaptive mesh
refinement (the cells at the interface of different materials have been refined
once). We see that instead of introducing hanging nodes, the mesh now contains 
pentagons and hexagons.
\begin{table}[H]
  \caption{Comparison of preconditioners for the AMR mesh}
  \begin{center}
    \begin{tabular}{|c|c|c|c|c|c|c|}
      \hline
       & No-DSA & CG & PCG-SGS & PCG-ML/U & PCG-ML/MIS & AGMG \\
      \hline
   SI iterations & 184     & 19      & 19       & 19      & 19       & 19 \\
Precond init (s) & NA      & NA      & 0.043463 & 0.358002 & 1.19301 & 0.0111\\
MIP calculation (s) & NA   & 48.1908 & 81.0992  & 25.2699 & 25.0699  & 
      2.56198\\
   CG iterations & NA      & 11300   & 4734     & 361     & 361      & 264 \\
     Total calculation (s) & 802.985 & 138.825 & 172.423  & 116.018 & 116.517  &
      94.1963\\
      \hline
    \end{tabular}
    \label{table_amr}
  \end{center}
\end{table}

\subsubsection{High aspect ratio grids}
%%%%%%%%%%%%%%%%%%%%%%%%%%%%%%%%%%%%%%%%%%%%%%%%%%%%%%%%%%%%%%%%%%%%%%%%%%%%%%%%%%

In these last two examples, a square domain  of $100cm \times 100cm$ with vacuum boundaries
is employed. There are 10,000 cells (thus, 40,000 degrees of freedom). Again, the relative 
tolerance on SI is $10^{-8}$ and the relative tolerance for CG is $10^{-10}$. 
An $S_8$ GLC angular quadrature is used. $\Sigma_t = 1cm^{-1}$, $\Sigma_s = 0.999cm^{-1}$,
and the source is $1n/(cm^3s)$. 
In the first run, the domain is discretized using 100 subdivisions of the $x$ and $y$
axes, i.e., 10,000 square cells with an aspect ratio of one. In the second run, 
the domain is discretized using 1,000 subdivision along $x$ and 10 along $y$ 
(the aspect ratio is then 100).
%
As expected, solving the MIP-DSA equations requires more CG iterations when the aspect
ratio increases. PCG-ML/U and PCG-ML/MIS are significantly more affected by the
increase in the aspect ratio than the other methods. AGMG is the least
affected by the change of aspect ratio and is again the best performing
preconditioner.
%
\begin{table}[!htbp]
  \caption{Comparison of preconditioners for a rectangular grid with an aspect
  ratio of 1}
  \begin{center}
    \begin{tabular}{|c|c|c|c|c|c|c|}
      \hline
       & No-DSA & CG & PCG-SGS & PCG-ML/U & PCG-ML/MIS & AGMG \\
      \hline
      SI iterations & 7311      & 21      & 21      & 21       & 21      & 21 \\
   Precond init (s) & NA        & NA      & 0.01422 & 0.051373 & 1.13144 &
      0.044 \\
MIP calculation (s) & NA        & 32.3825 & 73.8422 & 24.0707  & 25.0065 &
      1.7114 \\
      CG iterations & NA        & 8363    & 4853    & 376      & 375     &
      221\\
Total calculation (s) & 7356.96 & 56.8993 & 98.2609 & 50.1247  & 51.5396 &
      25.9306 \\
      \hline
    \end{tabular}
    \label{table_ar_1}
  \end{center}
\end{table}
%
\begin{table}[!htbp]
  \caption{Comparison of preconditioners for a rectangular grid with an aspect
  ratio of 100}
  \begin{center}
    \begin{tabular}{|c|c|c|c|c|c|c|}
      \hline
       & No-DSA & CG & PCG-SGS & PCG-ML/U & PCG-ML/MIS & AGMG \\
      \hline
      SI iterations & 7304    & 24      & 24        & 24       & 24      & 24 \\
   Precond init (s) & NA      & NA      & 0.0164239 & 0.362463 & 1.03128 & 0.052 \\
MIP calculation (s) & NA      & 372.227 & 742.902   & 941.06   & 922.258 &
      6.93176 \\
      CG iterations & NA      & 84802   & 43466     & 14180    & 13896   & 821 \\
Total calculation (s) & 9035.6 & 414.301 & 784.77   & 985.796  & 966.77  &
      44.7032 \\
      \hline
    \end{tabular}
    \label{table_ar_100}
  \end{center}
\end{table}                  
