\begin{abstract}
  The Modified Interior Penalty (MIP) Diffusion Synthetic Acceleration (DSA)
  is adapted to the PieceWise Linear Discontinuous (PWLD) finite elements.
  This provides a DSA preconditioner for arbitrary polygonal meshes. Arbitrary
  polygonal meshes can be used to model complex objects and to decrease the
  number of unknowns or to employ
  adaptive mesh refinement without the need to treat so-called ``hanging
  nodes''. Fourier analyses are 
  performed on the new MIP formulation and it is shown that this scheme is
  always stable with isotropic scattering, including high-aspect ratio cells.
  Numerical results are presented for three different unstructured meshes and 
  is shown to be 
  efficient on all these meshes. The system of equations produced by the MIP 
  is symmetric and positive definite and it is typical solved using
  Conjugate Gradient (CG) preconditioned with a Symmetric Gauss-Seidel
  scheme. In this research, we also compare CG preconditioned with 
  Symmetric Gauss-Seidel and CG preconditioned by Algebraic MultiGrid
  methods (AMG). AMG preconditioning is shown to be much more effective and
  efficient than SGS preconditioning.
\end{abstract}
