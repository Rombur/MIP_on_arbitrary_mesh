\begin{abstract}
  \textcolor{red}{The Modified Interior Penalty (MIP) Diffusion Synthetic Acceleration (DSA) technique
  is extended to the Piece-Wise Linear Discontinuous (PWLD) finite elements.\\
	In this paper, the Diffusion Synthetic Acceleration (DSA) technique, based
	on the Modified Interior Penalty (MIP), is adapted to Piece-Wise Linear 
	Discontinuous (PWLD) finite elements.}\\
  This \textcolor{red}{provides/yields/results in} an efficient DSA preconditioner for arbitrary 
	polygonal/polyhedral meshes. 
	Such arbitrary grids (which include triangular and tetrahedral meshes as a subset) can be used to model complex objects; 
	they can also be advantageously employed 
	with locally refined spatial grids without the need to deal with ``hanging nodes''. 
  The MIP technique was originally derived to solve diffusion problems using discontinuous finite elements;
	it yields a system of linear equations that is 
  is Symmetric Positive Definite (SPD). Thus, solution techniques, such as Preconditioned 
	Conjugate Gradient (PCG), can be effectively employed. 
	In this research, we compare Symmetric Gauss-Seidel (SGS) and Algebraic MultiGrid (AMG) methods
	as preconditioners
  Fourier analyses are 
  performed for the MIP-DSA formulation using PWLD finite elements and we show that this scheme is
  always stable and effective at reducing the spectral radius for iterative transport
  solves, including for grids with high-aspect ratio cells.
  Numerical results are presented for different grids (quadrilateral, hexagonal, polygonal, and rectangular
	with local mesh adaptation). 
  AMG preconditioning for the MIP system is shown to be significantly more efficient 
  than SGS preconditioning.
\end{abstract}
