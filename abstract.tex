\begin{abstract}
  The Modified Interior Penalty (MIP) Diffusion Synthetic Acceleration (DSA)
  is adapted to the PieceWise Linear Discontinuous (PWLD) finite elements.
  This allows MIP to be used on arbitrary polygonal meshes. Arbitrary
  polygonal meshes can be used to decrease the number of unknowns or to use
  adaptive mesh refinement without having hanging nodes. Fourier analysis is
  performed on the new MIP and it is shown that this scheme is always stable 
  if the scattering is isotropic and the mesh uses rectangular cells. The 
  method is tested on three different unstructured meshes and is shown to be 
  efficient on all these meshes. The system of equations produced by the MIP 
  are Symmetric and Positive Definite (SPD) and is customary solved using
  Conjugate Gradient (CG) preconditioned with Symmetric Successive
  OverRelaxation (SSOR). In this research, we compare CG preconditioned with 
  Symmetric Gauss-Seidel (SGS) and CG preconditioned by Algebraic MultiGrid
  methods (AMG). AMG preconditioning is shown to be much more effective and
  efficient than SGS preconditioning.
\end{abstract}
