\begin{abstract}
  %\textcolor{red}{The Modified Interior Penalty (MIP) Diffusion Synthetic Acceleration (DSA) technique
  %is extended to the Piece-Wise Linear Discontinuous (PWLD) finite elements.}
	In this paper, a Diffusion Synthetic Acceleration (DSA) technique
	is developed for Piece-Wise Linear Discontinuous (PWLD) finite element
	approximations on arbitrary polygonal grids.
%	
  The discretization of the DSA equations uses the Modified Interior Penalty (MIP), derived from a standard
	discontinuous finite element technique for the standard form of the diffusion equation.
  The MIP technique yields a system of linear equations that  
  is Symmetric Positive Definite (SPD). Thus, solution techniques, such as Preconditioned 
	Conjugate Gradient (PCG), can be effectively employed. 
	This results in an efficient DSA preconditioner for arbitrary 
	polygonal meshes. Such grids (which include triangular and tetrahedral meshes as a subset) can be used to model 
	complex objects; 	they can also be advantageously employed 
	with locally refined spatial grids without the need to deal with ``hanging nodes''. 
%
	Fourier analyses are 
  performed for the MIP-DSA formulation using PWLD finite elements, and we show that this scheme is
  always stable and effective at reducing the spectral radius for iterative transport
  solves, even for grids with high-aspect ratio cells.
  Numerical results are presented for different grids (quadrilateral, hexagonal, polygonal, and rectangular
	with local mesh adaptation). 
%
	Algebraic MultiGrid (AMG) and Symmetric Gauss-Seidel (SGS) are employed as conjugate gradient preconditioners for the MIP 
	system. AMG is shown to be significantly more efficient than SGS.
\end{abstract}
