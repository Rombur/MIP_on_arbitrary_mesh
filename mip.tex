%%%%%%%%%%%%%%%%%%%%%%%%%%%%%%%%%%%%%%%%%%%%%%%%%%%%%%%%%%%%%%%%%%%%%%%%%%%%%%%%%%
%%%%%%%%%%%%%%%%%%%%%%%%%%%%%%%%%%%%%%%%%%%%%%%%%%%%%%%%%%%%%%%%%%%%%%%%%%%%%%%%%%
\section{DSA on arbitrary polygonal cells} \label{sec_mip}
%%%%%%%%%%%%%%%%%%%%%%%%%%%%%%%%%%%%%%%%%%%%%%%%%%%%%%%%%%%%%%%%%%%%%%%%%%%%%%%%%%
%%%%%%%%%%%%%%%%%%%%%%%%%%%%%%%%%%%%%%%%%%%%%%%%%%%%%%%%%%%%%%%%%%%%%%%%%%%%%%%%%%

%%%%%%%%%%%%%%%%%%%%%%%%%%%%%%%%%%%%%%%%%%%%%%%%%%%%%%%%%%%%%%%%%%%%%%%%%%%%%%%%%%
\subsection{DSA Solution Principle}
%%%%%%%%%%%%%%%%%%%%%%%%%%%%%%%%%%%%%%%%%%%%%%%%%%%%%%%%%%%%%%%%%%%%%%%%%%%%%%%%%%

As noted earlier, in thick diffusive configurations, standard iterative techniques 
applied to transport solves 
can be slowly converging. DSA can be employed to accelerate their convergence.
The idea behind synthetic acceleration is that the error between the (yet
unknown) transport solution and the current iterate can be estimated from a
computationally less expensive process, yielding a correction term to be added
to the current iterate in order to improve the next iterate. In DSA, a
corrective scalar flux contribution, $\delta \Phi$, is sought through the following 
diffusion solve, 
\begin{equation}
  \bs{A}\ \delta \Phi = \bs{\Sigma}\(\Phi^{(\ell+1/2)} - \Phi^{(\ell)}\).
\end{equation}
where the source term is a scattering term due to the
difference between the previous iterate scalar flux $\Phi^{(\ell)}$ and the
newest scalar flux, $\Phi^{(\ell+1/2)}$, obtained after a single transport sweep. 
The next scalar flux iterate is obtained by adding the scalar flux correction to
the scalar flux obtained after a transport sweep:
\begin{equation}
  \Phi^{(\ell+1)} = \Phi^{(\ell+1/2)}+\delta \Phi.
\end{equation}
$\bs{A}$ is the DSA matrix.
Ideally, $\bs{A}$ should be SPD (such that efficient iterative techniques can 
be employed in its linear solve) and easy to form (even on arbitrary grids).

%%%%%%%%%%%%%%%%%%%%%%%%%%%%%%%%%%%%%%%%%%%%%%%%%%%%%%%%%%%%%%%%%%%%%%%%%%%%%%%%%%
\subsection{Modified Interior Penalty DSA scheme for polygons}
%%%%%%%%%%%%%%%%%%%%%%%%%%%%%%%%%%%%%%%%%%%%%%%%%%%%%%%%%%%%%%%%%%%%%%%%%%%%%%%%%%

Now, we discuss the Modified Interior Penalty as a DSA scheme for arbitrary polygons. 
This DSA scheme employs the same discontinuous finite elements used in the 
transport discretization for the discretization of the diffusion operator. 
MIP-DSA has been shown to be always stable for isotropic scattering on triangular 
cells \cite{mip}. The MIP diffusion solve is based on the standard Interior Penalty scheme (IP)
\cite{ip}, with an \textcolor{red}{appropriately modified} penalty term. Consider 
following diffusion update equation:
\begin{align}
  \label{eq:dsa}
  \left(-\bn \cdot \mathrm{D} \bn  + \Sigma_a \right) \delta \phi &= Q_0 &\textrm{ for }\br \in
  \mc{D}\\
  \label{eq:dsa-bc}
  \frac{1}{4}\delta \phi - \frac{1}{2} \mathrm{D} \partial_n \delta \phi & = 0 &\textrm{ for }
  \br \in \partial \mc{D} %^d \\
  %-\mathrm{D} \partial_n \phi &= J^{inc} & \textrm{ for } \br \in \partial
  %\mc{D}^r,
\end{align}
where $\mathrm{D}$ is the diffusion coefficient and $\Sigma_a$ is the absorption cross section. 
%
%$\partial \mc{D}^d$ is the
%boundary of the domain with Dirichlet condition, $\partial \mc{D}^r$ is the
%boundary of the domain with reflective condition, 
%
For DSA, the volumetric source term is given by the scattering due to the difference in scalar flux
between two iterations, $Q_0=\Sigma_s\(\Phi^{(\ell+1/2)}-\Phi^{(\ell)}\)$, 
%
%$J^{inc} = \sum_{\bo_m\cdot\bs{n}_b >0} w_m 
%|\bo_m \cdot \bs{n}_b| \delta \psi_d$, 
%
and the known incoming angular flux from the transport problem translates into the diffusion
vacuum boundary condition given in \eqt{eq:dsa-bc}, where $\partial_{n} = \bs{n}\cdot \bn$
and $\bs{n}$ is the outer normal unit vector on the domain's boundary, $\partial \mc{D}$. 
%
%where
%$\bs{n}_e$ is the normal unit vector associated with a given edge $e$ (on the
%boundary $\bs{n}_e = \bs{n}_b$). 
%The only difference between IP and MIP is the
%penalty factor $\kappa_e^{IP}$ used for IP is replaced by a new penalty factor
%$\kappa_e^{MIP}$. Therefore, 
%
The weak form of MIP-DSA equations is given by:
\begin{equation}
a(\delta \phi,\phi^*) = l(\phi^*)
\label{mip}
\end{equation}
%
with the bilinear (matrix) form:
\begin{equation}
\begin{split}
a(\delta \phi,\phi^*) =& \(\Sigma_a \delta \phi,\phi^*\)_{\mc{D}}+
  (\mathrm{D}\bn\delta \phi,\bn\phi^*)_{\mc{D}} + \(\kappa_e^{MIP} \llb\delta \phi\rrb,
\llb\phi^*\rrb\)_{E_h^i} + \(\llb\delta \phi\rrb,\ldb\mathrm{D}\partial_n
\phi^*\rdb\)_{E_h^i} +\\
& \(\ldb\mathrm{D}\partial_n \delta \phi\rdb,\llb\phi^*\rrb\)_{E_h^i} +
\(\kappa_e^{MIP}
\delta \phi,\phi^*\)_{\partial \mc{D}^d} -\frac{1}{2} \(\delta \phi,\mathrm{D} \partial_n
\phi^*\)_{\partial \mc{D}^d} - \frac{1}{2}\(\mathrm{D}\partial_n
\delta \phi,\phi^*\)_{\partial \mc{D}^d}
\label{mip_b}
\end{split}
\end{equation}
%
and the linear (right-hand-side) form
%
\begin{equation}
l(\phi^*) = (Q_0,\phi^*)_{\mc{D}}\ . % + (J^{inc},\phi^*)_{\partial \mc{D}^r} .
\label{mip_l}
\end{equation}
%
The notations used are as follows:
the domain integral is split onto the element integrals, $(f,g)_{\mc{D}} = \sum_{K\in \mathbb{T}_h} \(f,g\)_K$, with the element integrals defined as 
$(f,g)_K = \int_K fg\ d\br$; 
the integral over all interior edges is $(f,g)_{E_h^i}=\sum_{e\in E_h^i}(f,g)_e$, with the edge integrals defined as 
$(f,g)_e = \int_e fg\ ds$. In the above, 
 $\mathbb{T}_h$ denotes the partition of domain
$\mc{D}$ into non-overlapping elements $K$, $E_h^i$ is the set of interior
edges, $\delta \phi$ is the numerical solution for the diffusion correction and $\phi^*$ is any test function.
Any entry $A_{ij}$ of matrix $\bs{A}$ is simply: $A_{ij}= a(b_j,b_i)$. The resulting matrix $\bs{A}$ is SPD
and thus can be solved using a preconditioned conjugate gradient technique.
%
%$\in W_{\mc{D}}^h$, $W_{\mc{D}}^h=\{P \in
%L^2(\mc{D}); P|_{K}\in V_p(K), \forall K \in \mathbb{T}_h\}$, where $V_p(K)$
%is the space of polynomials of degree up to $p$ on element $K$,  
%
The jump and mean value of a variable $u$  at the interface between two elements is defined as follows:
\begin{equation}
\llb u \rrb = u^+ - u^- \text{ and } \ldb u\rdb = \frac{u^+ + u^-}{2} ,
\end{equation}
respectively.
%
The definition of the left/right values along a given edge $e$ is
\begin{equation}
u^{\pm}= \lim_{s\rightarrow 0^{\pm}}u(\bs{r}+s\bs{n}_e) ,
\end{equation}
where $\bs{n}_e$ is the normal unit vector associated with the given edge $e$
(on the boundary $\bs{n}_e$ is the external normal).\\
%
The MIP penalty coefficient is given by
\begin{equation}
\kappa_e^{MIP} = \max\(\kappa_e^{IP},\frac{1}{4}\)
\end{equation}
with:
\begin{equation}
\kappa_e^{IP} = \left\{
\begin{aligned}
&\frac{c}{2} \frac{\mathrm{D^+}}{h_{\bot}^+} + \frac{c}{2}
\frac{\mathrm{D}^-}{h_{\bot}^-} & \textrm{on interior edges, i.e., }
e\in E_h^i\\
&c \frac{\mathrm{D}}{h_{\bot}} & \textrm{on boundary edges, i.e., } e
\in\partial \mc{D} %^d 
\end{aligned}
\right. 
\end{equation}
where $c$ is a constant (chosen equal to 4 here) and $h_{\bot}$ is the length of the cell in the direction
orthogonal to the edge $e$. 
%
%MIP yields only a correction for the scalar flux
%but by assuming that the angular dependence satisfies a diffusion expansion,
%the angular correction can be computed using:
%\begin{equation}
  %\varepsilon_d = \frac{1}{4\pi}(\phi-3\mathrm{D}\bn\phi\cdot \bo_d)
%\end{equation}
%This correction can be used when some of the boundary conditions are periodic
%or reflective.
%
When polygonal cells are employed, there is no 
simple way to compute $h_{\bot}$. To estimate $h_{\bot}$, we 
assume that the polygonal cells do not deviate too much from regular polygons. 
In such case, if the cell has an even number of edges, the orthogonal 
length equals two times the apothem, i.e., two times the segment between the 
midpoint of a side of the polygon and the center of this polygon 
$\(\textrm{apothem}=2\times \frac{\textrm{area}}{\textrm{perimeter}}\)$. 
\textcolor{red}{I think you mean area and perimeter of the triangle formed by
the edge $e$ and polygon the mid-point. Confirm. ????}
If the cell has an odd number of edges, the orthogonal length is given by the 
apothem plus the circumradius, i.e., the radius of the circle circumscribed to 
the polygon $\(\textrm{circumradius}=\sqrt{\frac{2\times \textrm{area}}{N_V
\sin\(\frac{2\pi}{N_V}\)}}\)$. A summary of the definitions used for $h_{\bot}$ for 
any polygon as a function of the number of vertices is given in
\Cref{table_h_bot}.
\begin{table}[H]
  \begin{center}
    \caption{Orthogonal length $h_{\bot}$ for different polygon types.}
    \begin{tabular}{|c|c|c|c|c|}
      \hline
      Number of vertices & 3 & 4 & $> 4$ and even & $> 4$ and odd \\
      \hline
      $h_{\bot}$ & $2 \times \frac{\textrm{area}}{L_e}$ &
$\frac{\textrm{area}}{L_e}$ & $4\times
\frac{\textrm{area}}{\textrm{perimeter}}$ & $2 \times
      \frac{\textrm{area}}{\textrm{perimeter}}+\sqrt{\frac{2\times
      \textrm{area}}{N_V\sin\(\frac{2\pi}{N_V}\)}}$\\
      \hline
    \end{tabular}
    \label{table_h_bot}
  \end{center}
\end{table}
