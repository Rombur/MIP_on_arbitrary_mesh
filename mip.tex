\section{MIP on arbitrary polygonal cells} \label{sec_mip}
\subsection{Solution Principle}
In this Section, we recall the MIP technique and extend it to arbitrary
polygonal/polyhedral cells. As noted earlier, in thick diffusive
configurations, standard iterative techniques applied to transport solves 
can be slowly converging. DSA can be employed to accelerate their convergence.
The idea behind synthetic acceleration is that the error between the (yet
unknown) transport solution and the latest iterate can be estimated from a
computationally less expensive process, yielding a correction term to be added
to the latest iterate in order to improve the next iterate. In DSA, the
corrective scalar flux contribution is sought through the following
diffusion solve, where the source term is a scattering term due to the
difference between the previous iterate scalar flux $\Phi^{(\ell)}$ and the
newest scalar flux, obtained after a transport sweep, $\Phi^{(l+1/2)}$. The
$(l+1/2)$ index indicates that the next iterate, $\Phi^{(l+1)}$ has not yet
been fully obtained after the transport sweep, but will be diffusion-corrected:
\begin{equation}
  \bs{A}\ \delta \Phi = \bs{\Sigma}\(\Phi^{(\ell+1/2)} - \Phi^{(\ell)}\).
\end{equation}
where $\bs{A}$ is the DSA matrix.
Then, the next iterate is obtained as follows:
\begin{equation}
  \Phi^{(\ell+1)} = \Phi^{(\ell+1/2)}+\delta \Phi.
\end{equation}
Ideally, $\bs{A}$ should be SPD (such that efficient iterative techniques can 
be employed in its linear solve) and easy to form (even on arbitrary grids).
\subsection{Modified Interior Penalty DSA}
Now, we can discuss the Modified Interior Penalty as a DSA scheme \cite{mip}. 
This DSA scheme employs the same discontinuous finite elements used in the 
transport discretization for the discretization of the diffusion operator. 
MIP-DSA has been shown to be always stable for isotropic scattering on triangular 
cells. It is based on Interior Penalty scheme (IP). This method applies
directly the IP technique to the following diffusion update equation:
\begin{align}
  -\bn \mathrm{D} \bn \phi + \Sigma_a \phi &= Q_0 &\textrm{ for }\br \in
  \mc{D}\\
  \frac{1}{4}\phi - \frac{1}{2} \mathrm{D} \partial_n \phi & = 0 &\textrm{ for }
  \br \in \partial \mc{D}^d\\
  -\mathrm{D} \partial_n \phi &= J^{inc} & \textrm{ for } \br \in \partial
  \mc{D}^r,
\end{align}
where $\mathrm{D}$ is the diffusion coefficient, $\partial \mc{D}^d$ is the
boundary of the domain with Dirichlet condition, $\partial \mc{D}^r$ is the
boundary of the domain with reflective condition, $Q_0=\Sigma_s
\(\Phi^{(\ell+1/2)}-\Phi^{(\ell)}\)$, $J^{inc} = \sum_{\bo_d\cdot\bs{n}_b >0} w_d 
|\bo_d \cdot \bs{n}_b| \delta \psi_d$, and $\partial_n = \bs{n}_e\cdot \bn$ where
$\bs{n}_e$ is the normal unit vector associated with a given edge $e$ (on the
boundary $\bs{n}_e = \bs{n}_b$). The only difference between IP and MIP is the
penalty factor $\kappa_e^{IP}$ used for IP is replaced by a new penalty factor
$\kappa_e^{MIP}$. Therefore, the weak form of MIP is given by:
\begin{equation}
b(\phi,\phi^*) = l(\phi^*)
\label{mip}
\end{equation}
with:
\begin{equation}
\begin{split}
b(\phi,\phi^*) =& \(\Sigma_a \phi,\phi^*\)_{\mc{D}}+
  (\mathrm{D}\bn\phi,\bn\phi^*)_{\mc{D}} + \(\kappa_e^{MIP} \llb\phi\rrb,
\llb\phi^*\rrb\)_{E_h^i} + \(\llb\phi\rrb,\ldb\mathrm{D}\partial_n
\phi^*\rdb\)_{E_h^i} +\\
& \(\ldb\mathrm{D}\partial_n \phi\rdb,\llb\phi^*\rrb\)_{E_h^i} +
\(\kappa_e^{MIP}
\phi,\phi^*\)_{\partial \mc{D}^d} -\frac{1}{2} \(\phi,\mathrm{D} \partial_n
\phi^*\)_{\partial \mc{D}^d} - \frac{1}{2}\(\mathrm{D}\partial_n
\phi,\phi^*\)_{\partial \mc{D}^d}
\label{mip_b}
\end{split}
\end{equation}
\begin{equation}
l(\phi^*) = (Q_0,\phi^*)_{\mc{D}}+ (J^{inc},\phi^*)_{\partial \mc{D}^r},
\label{mip_l}
\end{equation}
where $(f,g)_{\mc{D}} = \sum_{K\in \mathbb{T}_h} \(f,g\)_K$, 
$(f,g)_K = \int_K fg\ d\br$, $(f,g)_{E_h^i}=\sum_{e\in E_h^i}(f,g)_e$, 
$(f,g)_e = \int_e fg\ ds$,
 $\mathbb{T}_h$ is the mesh used to discretize the domain
$\mc{D}$ into nonoverlapping elements $K$, $E_h^i$ is the set of interior
edges, $\phi$ and $\phi^* \in W_{\mc{D}}^h$, $W_{\mc{D}}^h=\{P \in
L^2(\mc{D}); P|_{K}\in V_p(K), \forall K \in \mathbb{T}_h\}$, where $V_p(K)$
is the space of polynomials of degree up to $p$ on element $K$, $\Sigma_a$ is 
the absorption cross section, $\llb\phi\rrb = \phi^+ - \phi^-$ is the jump of 
$\phi$ at the interface between two elements, 
$\ldb\phi\rdb = \frac{\phi^+ + \phi^-}{2}$ is the mean of $\phi$ 
at the interface between two elements, 
$\phi^{\pm}=\lim_{s\rightarrow 0^{\pm}}\phi(\bs{r}+s\bs{n}_e)$, and
$\kappa_e^{MIP} = \max\(\kappa_e^{IP},\frac{1}{4}\)$
with:
\begin{equation}
\kappa_e^{IP} = \left\{
\begin{aligned}
&\frac{c(p^+)}{2} \frac{\mathrm{D^+}}{h_{\bot}^+} + \frac{c(p^-)}{2}
\frac{\mathrm{D}^-}{h_{\bot}^-} & \textrm{on interior edges, i.e., }
e\in E_h^i\\
&c(p) \frac{\mathrm{D}}{h_{\bot}} & \textrm{on boundary edges, i.e., } e
\in\partial \mc{D}^d 
\end{aligned}
\right. 
\end{equation}
where $c(p)$ is given by $c(p) = 2p (p+1)$, $p$ is the polynomial order ($p=1$
in this paper) and $h_{\bot}$ is the length of the cell in the direction
orthogonal to the edge $e$. MIP yields only a correction for the scalar flux
but by assuming that the angular dependence satisfies a diffusion expansion,
the angular correction can be computed using:
\begin{equation}
  \varepsilon_d = \frac{1}{4\pi}(\phi-3\mathrm{D}\bn\phi\cdot \bo_d)
\end{equation}
This correction can be used when some of the boundary conditions are periodic
or reflective.

When PWLD finite elements are used instead of LD finite elements, the weak form 
of MIP remains unchanged. But for general polygons, there is no 
simple way to compute $h_{\bot}$. To estimate $h_{\bot}$, we 
assume that the polygonal cells are not too far from being regular polygonal 
cells. In such case, if the cell has an even number of edges, the orthogonal 
length equals two times the apothem, i.e., two times the segment between the 
midpoint of a side of the polygon and the center of this polygon 
$\(\textrm{apothem}=2\times \frac{\textrm{area}}{\textrm{perimeter}}\)$. If 
the cell has an odd number of edges, the orthogonal length is given by the 
apothem plus the circumradius, i.e., the radius of the circle circumscribed to 
the polygon $\(\textrm{circumradius}=\sqrt{\frac{2\times \textrm{area}}{N_V
\sin\(\frac{2\pi}{N_V}\)}}\)$. Therefore, $h_{\bot}$ is given in
\Cref{table_h_bot}.
\begin{table}[H]
  \begin{center}
    \caption{Orthogonal length of the cell for different cells.}
    \begin{tabular}{|c|c|c|c|c|}
      \hline
      Number of edges & 3 & 4 & $> 4$ and even & $> 4$ and odd \\
      \hline
      $h_{\bot}$ & $2 \times \frac{\textrm{area}}{L_e}$ &
$\frac{\textrm{area}}{L_e}$ & $4\times
\frac{\textrm{area}}{\textrm{perimeter}}$ & $2 \times
      \frac{\textrm{area}}{\textrm{perimeter}}+\sqrt{\frac{2\times
      \textrm{area}}{N_V\sin\(\frac{2\pi}{N_V}\)}}$\\
      \hline
    \end{tabular}
    \label{table_h_bot}
  \end{center}
\end{table}
