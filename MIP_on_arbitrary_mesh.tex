\documentclass{article}
\usepackage{amsmath}
\usepackage{array}
\usepackage{xspace}
\usepackage{color}
\usepackage{graphicx}
\usepackage{float} % utiliser H pour forcer a mettre l'image ou on veut
\usepackage{lscape} % utilisation du mode paysage
\usepackage{mathbbol} % permet d'avoir le vrai symbol pour les reels grace a mathbb
\usepackage{enumerate} % permet d'utiliser enumerate
\usepackage{marvosym} % permet d'avoir le symbol pour le nucleaire
\usepackage{moreverb} % permet d'utiliser verbatimtab : conservation la tabulation
\usepackage{stmaryrd} % permet d'utiliser \llbrackedt et \rrbracket : double crochet
\usepackage{caption}
\usepackage{subcaption}
\usepackage[noabbrev]{cleveref} % permet d'utiliser cref and Cref

\usepackage{setspace}
%\doublespacing

\setlength{\textwidth}{16.6cm}
\setlength{\textheight}{21cm}
\setlength{\oddsidemargin}{0cm}
\setlength{\headsep}{5pt} 

\newcommand\bn{\boldsymbol{\nabla}}
\newcommand\bo{\boldsymbol{\Omega}}
\newcommand\br{\mathbf{r}}
\newcommand\la{\left\langle}
\newcommand\ra{\right\rangle}
\newcommand\bs{\boldsymbol}
\newcommand\red{\textcolor{red}}
\newcommand\ldb{\{\!\!\{}
\newcommand\rdb{\}\!\!\}}
\newcommand\llb{\llbracket}
\newcommand\rrb{\rrbracket}
\newcommand\mc{\mathcal}

\renewcommand{\(}{\left(}
\renewcommand{\)}{\right)}
\renewcommand{\[}{\left[}
\renewcommand{\]}{\right]}

\newcommand{\sn}{\ensuremath{S_n}\xspace}
\newcommand{\tf}{b_i}

\newtheorem{algorithm}{Algorithm}[section]

% DGFEM commands
\newcommand{\jmp}[1]{\llb #1 \rrb}                     % jump
\newcommand{\mvl}[1]{\ldb #1 \rdb}             % mean value
%


\begin{document}
\title{Discontinuous Diffusion Synthetic Acceleration for Sn Transport on
2D Arbitrary Polygonal Meshes}
\author{} 
\date{}
\maketitle

\begin{abstract}
  The Modified Interior Penalty (MIP) Diffusion Synthetic Acceleration (DSA) technique
  is extended to the Piece-Wise Linear Discontinuous (PWLD) finite elements.
  This \textcolor{red}{provides} a DSA preconditioner for arbitrary polygonal/polyhedral meshes. 
	Arbitrary grids can be used to model complex objects and can also be advantageously employed 
	within locally refined spatial grid without the need to treat so-called ``hanging
  nodes''. 
  The MIP technique yields a system of linear equations that is 
  is symmetric and positive definite. This system is typically solved using
  Conjugate Gradient (CG) preconditioned with a \textcolor{red}{Symmetric Gauss-Seidel}
  (SGS) scheme. In this research, we also compare SGS-preconditioned CG with 
  CG preconditioned using an Algebraic MultiGrid  method (AMG). 
  Fourier analyses are 
  performed for the PWLD MIP formulation and we show that this scheme is
  always stable and effective at reducing the spectral radius for iterative transport
  solves, including for grids with high-aspect ratio cells.
  Numerical results are presented for three different unstructured grids. 
  AMG preconditioning for the MIP system is shown to be significantly more efficient 
  than SGS preconditioning.
\end{abstract}

\section{Introduction}
Several discretization methods haven been developed for arbitrary polygonal
meshes \cite{pwld_2d,pwld_3d,pwl_diffusion,palmer_ane,palmer_proc,palmer_fe,
wachspress,cell_centered_diff,mimetic}. Polygonal cells can be advantageous 
because the number of unknowns can be reduced while maintaining symmetry 
within the mesh. This potential reduction in the number of unknowns can be 
seen by comparing a hexagonal cell with a triangular discretization of the 
same area:
\begin{figure}[H]
\centering
\includegraphics[width=0.5\textwidth]{hex_tri_cells}
\caption{Hexagonal cell versus triangle cells.}
\end{figure}
If there is one unknown per vertex, the hexagonal cell will have only six
unknowns compared to the twelve unknowns of the triangular discretization. 
Another advantage of polygonal cells is that they can be used for adaptive 
mesh refinement (AMR) \cite{amr_rad,amr_block,amr_unstruc} without having to
deal with hanging nodes \cite{arbitrary_hanging_nodes,dealII_hanging_nodes,
locally_hanging_nodes}. On the figure below, the left cell is a pentagon whereas 
the two cells on the right are quadrilaterals:
\begin{figure}[H]
\centering
\includegraphics[width=0.3\textwidth]{amr}
\caption{AMR mesh.}
\end{figure}
Among the different discretization schemes for polygonal meshes, the PieceWise 
Linear Discontinuous (PWLD) finite elements \cite{pwld_2d,pwld_3d} have been 
successfully used to solve the transport equation. Another variant of the
PieceWise Linear finite elements, the PieceWise Linear Continuous (PWLC) finite 
elements were also developed and used to discretize the diffusion equation. 
They have shown to be second order and to produce a symmetric positive (SPD) 
matrix contrarily to others discretization for arbitrary polygonal meshes 
\cite{pwl_diffusion}. However, no Diffusion Synthetic Acceleration (DSA) 
\cite{dsa_ref,larsen_dsa,consistent_p1,mip} was developed using either PWLD or
PWLC finite elements. In this work, we will remedy this lack by adapting the 
Modified Interior Penalty (MIP) DSA developed in \cite{mip} for triangular 
cells to PWLD finite elements which will allow the DSA scheme to work on arbitrary 
polygonal cells. Since MIP produces SPD equations, it has usually 
been solved using conjugate gradient (CG) preconditioned by SSOR. In this
paper, the effectiveness of algebraic multigrid methods (AMG) to precondition 
the Krylov solver \cite{amg,amg_course} will be tested. Algebraic multigrid methods 
allow the use of multigrid techniques when there is no grid or when the mesh is 
unstructured. Instead of using a succession of grids based on the geometry of the 
problems, the grids are based on properties of the matrix.

The remainder of this paper is organized as follows. In Sec.(\ref{sec_mip}),
we introduce the PWLD finite elements and we adapt MIP to this discretization. 
In Sec.(\ref{sec_amg}), we briefly explain AMG and we introduce the two different 
AMG methods that we will use, the ML package of Trilinos \cite{ml_guide} and the 
AGMG (AGgregation-based algebraic MultiGrid) code \cite{agmg_guide}. In 
Sec.(\ref{sec_res}), we show the Fourier analysis of MIP discretized with PWLD
and we compare the different preconditioners of the CG solver. In 
Sec.(\ref{sec_conc}), we give our conclusions.

\section{Discretization and Solution Techniques for the $S_n$ Transport
Equation}\label{sec_transport}
Here, we briefly recall the \sn transport equations, the iterative techniques
employed to solve them, and discuss the PWLD discontinuous spatial
discretization for the transport equation with an emphasis on arbitrary
polygonal/polyhedral grids.
\subsection{The \sn Transport Equations}
Given an angular quadrature set $\{\bo_d,w_d\}_{1\leq d\leq M}$, the one-group
\sn transport equation with isotropic source and scattering is:
\begin{equation}
  \(\bo_d\cdot \bn + \Sigma_t(\br)\)\psi_d (\br) = \frac{1}{4\pi} \Sigma_s
  (\br) \phi(\br) + \frac{1}{4\pi} S (\br), \ \textrm{ for } \br \in \mc{D},\
  1 \leq d \leq M
  \label{transport_sn}
\end{equation}
with $\psi_d(\br) = \psi(\br,\bo_d)$ the angular flux at position $\br$ in
direction $\bo_d$, $\Sigma_t$ and $\Sigma_s$ the total and scattering cross
sections, respectively, and $\mc{D}$ the spatial domain. The scalar flux is
defined as and evaluated as follows:
\begin{equation}
  \phi(\br) \equiv \int_{4\pi} \psi(\br,\bo) d\bo \approx \sum_{d=1}^M w_d
  \psi_d (\br).
\end{equation}
For brevity, we assume only incoming boundary conditions, $\psi_d(\br_b) =
\psi_d^{inc}(\br_b)$, for any boundary location $\br_b \in \partial \mc{D}_d^-
= \{\partial \mc{D} \textrm{ such that }\bo_d \cdot \bs{n}_b <0\}$, where
$\bs{n}_b = \bs{n}(\br_b)$ is the outward unit normal vector at the boundary. 

\Cref{transport_sn} can be written in a compact form using operators:
\begin{align}
  & \bs{L} \Psi = \bs{M \Sigma}\Phi + S \equiv q, \label{L_Psi}\\
  &\Phi = \bs{D} \Psi, \label{Phi}
\end{align}
where $\Psi$ is the vector of angular fluxes, $\Phi$ the vector scalar flux,
$q$ is the total (scattering+external) source $\bs{L}$ is the streaming
operator, $\bs{M}$ is the moment-to-direction operator, and $\bs{D}$ is the
direction-to-moment operator. $\bs{L} = diag
(\bs{L}_1,\hdots,\bs{L}_d,\hdots,\bs{L}_M)$ is a diagonal operator; given a
total cross section, one can solve independently for the resulting angular
fluxes in all directions. The action of $\bs{L}^{-1}$ is often referred to as
a \emph{transport sweep} when discontinuous spatial approximations are
employed because, for any direction $\bo_d$, the action of $\bs{L}_d^{-1}$ can
be obtained by traversing the mesh (i.e., sweeping) in a specific ordering of
the cells, thus one needs only to solve a small linear system of equations,
cell by cell. The order in which the elements are solved constitutes the graph
of the sweep; for brevity and because this is not the focus of this article,
we do not expand on situations where the graph can present some dependencies
(cycles); in such a case, these dependencies can either be lagged within the
iterative procedure or the solution vector consisting of the scalar flux is
augmented by the angular fluxes that cause the cycle.
\subsection{Solution Techniques}
\Cref{L_Psi,Phi} can be solved using the Source Iteration (SI) method (a
stationary iterative technique also knows as Richardson iteration). The SI
techniques at the $\ell^{th}$ iteration is given by :
\begin{equation}
  \Phi^{(\ell+1)} = \bs{DL}^{-1} \(\bs{M\Sigma}\Phi^{(\ell)} + S\)
\end{equation}
Alternatively, a subspace Krylov method (usually GMRes) can be employed to
solve the system of equations:
\begin{equation}
  \(\bs{I} - \bs{DL}^{-1}\bs{M \Sigma}\) \Phi = \bs{DL}^{-1}S
\end{equation}
Both the SI and the GMRES approaches require transport sweeps (the action of
$\bs{L}^{-1}$ is required in both procedures).

When the scattering ratio $c=\frac{\Sigma_s}{\Sigma_t}$ tends to one in
optically thick domains, the number of SI and GMRES can become large. To
accelerate the convergence, a DSA preconditioner is needed; the MIP
discontinuous finite element discretization of the diffusion equation for
arbitrary polygonal/polyhedral grids is presented in \Cref{sec_mip}.
\subsection{Discontinuous Finite Element Discretization on Arbitrary Grids}
The domain $\mc{D}$ is meshed into element $K$ (specifically polygons and
polyhedra). For a given streaming direction $\bo_d$, the discontinuous finite
element scheme on a given element $K$ is given by:
\begin{equation}
  -\int_{K} \(\psi_d \bo_d \cdot \bn b + \Sigma_t \Psi_d b \)\ d\br +
  \int_{\partial K^+} \bo_d \cdot \bs{n} \Psi_d b\ d\br = \int_{K} qb\ d\br +
  \int_{\partial K^{-}} |\bo_d \cdot \bs{n}| \psi_d^{\uparrow}b \ d\br
  \label{transport_int}
\end{equation}
where $b$ represents a generic basis function, $\partial K^{-}$ is the inflow
face of element $K$, $\partial K^{+}$ is the outflow face of element $K$. The
angular flux values on an inflow face, denoted by $\psi_d^{\uparrow}$ in
\cref{transport_int}, are taken from the upwind neighbor element of that face.

Next, we define the basis function $b$ for the PWLD method. First, we
introduce a within-cell point $c$ inside the polygon in 2D (resp., polyhedron
in 3D). The coordinates of $c$ are weighted averages of the vertex
coordinates:
\begin{equation}
  u_c = \sum_{j=1}^{N_V} \alpha_j u_j
\end{equation}
where $u=x$, $y,$ or $z$, $N_V$ is the number of vertices for the cell under
consideration, and the weights are such that $\sum_{j=1}^V \alpha_j =1$,
$\alpha_j \geq 0\ \forall j$. In 2D, the basis function at vertex $j$ is
defined by \cite{pwld_2d}:
\begin{equation}
  b_j(x,y) = t_j(x,y) + \alpha_j t_c(x,y)
\end{equation}
where $t_j(x,y)$ is a linear function such that $t_j(x,y)$ is unity at vertex
$j$ and zero at $j-1$, $j+1$, and $c$. The $t_c(x,y)$ function is a ``hat''
function in the interior of the cell, is unity at point $c$, and zero at all
the vertices of the cell. In 3D, the same definition applies for $t_c$ but
$t_j$ is now constructed on a tetrahedron obtained from two adjacent vertices,
the cell center point $c$, and a face center point, where that face contains
both vertices. The PWLD method employs a notion of sub-cells on each arbitrary 
polygon (polyhedron); the number of PWLD basis functions is equal to the
number of vertices in the polygon (polyhedron). In this paper, the arbitrary
positive weights $\alpha_j$ are chosen to be $\frac{1}{N_V}$, thus, on a
quadrilateral cell, one has $\alpha_j =0.25 \forall j$. Finally, note that on
triangular (resp., tetrahedral) cells, the PWLD basis functions reduce to the
standard Linear Discontinuous (LD) basis functions of $\alpha_j = \frac{1}{3}$
$\(\textrm{resp., }\alpha_j = \frac{1}{4}\)$. Given the definition of the PWLD
finite elements, it may seem complicated to build the mass and the gradient
matrices on an arbitrary polygonal cell but the construction of such matrices
can be greatly simplified using the sub-cells. By looping over all of the
sub-cells of a cell, the elementary matrices of the cell can easily be
computed.

\section{MIP on arbitrary polygonal cells} \label{sec_mip}
In this section, we introduce the PieceWise Linear Discontinuous (PWLD) finite
elements developed in \cite{pwld_2d,pwld_3d,pwl_diffusion}. To obtain PWLD
basis functions for two-dimensional polygons, we first divide each polygonal
cell into ``side'' sub-cells. A ``side'' sub-cell is a triangle made from two
adjacent vertices and the within-cell point $c$. The coordinates of $c$ are weighted
averages of the vertex coordinates:
\begin{align}
& x_c = \sum_{j=1}^{V} \alpha_{j} x_j\\
& y_c = \sum_{j=1}^{V} \alpha_{j} y_j
\end{align}
where $\sum_{j=1}^{V} \alpha_{j}=1$ and $V$ is the number of vertices of the cell.\\
The basis function at vertex $j$ is defined as \cite{pwld_2d}:
\begin{equation}
b_{j} (x,y) = t_{j}(x,y) + \alpha_j t_c(x,y)
\end{equation}
where the $t_j$ functions are standard linear functions on triangles: $t_j
(x,y)$ is unity at vertex $j$, zero at $j-1$, $j+1$ and $c$, and zero in all
triangular sides that are not in contact with vertex $j$. The function $t_c
(x,y)$ is unity at $c$ and zero at each vertex. The parameters $\alpha_{j}$
are arbitrary positive weights. In this paper, we use 
$\alpha_{j}=\frac{1}{V}$. On a square cell with $\alpha_{j}=0.25\ \forall j$, 
the basis functions are given in Figure (\ref{pwld}):
\begin{figure}[H]
\centering
\subfloat[First basis function]{\includegraphics[width=0.3\textwidth]{pwld_1}}
\subfloat[Second basis function]{\includegraphics[width=0.3\textwidth]{pwld_2}}\\
\subfloat[Third basis function]{\includegraphics[width=0.3\textwidth]{pwld_3}}
\subfloat[Fourth basis function]{\includegraphics[width=0.3\textwidth]{pwld_4}}
\caption{PWLD basis function}
\label{pwld}
\end{figure}
On triangular cells, the PWLD basis functions reduces to the standard LD basis 
functions. Since the PWLD basis functions are built on triangular ``side'' 
sub-cells, the integrals over the area of a cell is replaced by a sum of integrals 
over the sides of the cell. 

After defining the spatial discretization that we employ here, we can now
define the Modified Interior Penalty DSA \cite{mip}. This DSA scheme uses 
discontinuous Galerkin finite elements for the spatial discretization and has
been shown to be always stable for isotropic scattering on triangular cells. 
The MIP weak form can be written as:
\begin{equation}
b(\phi,\phi^*) = l(\phi^*)
\label{mip}
\end{equation}
with:
\begin{equation}
\begin{split}
b(\phi,\phi^*) =& \(\Sigma_a \phi,\phi^*\)_{\mc{D}}+
(\mathrm{D}\bn\phi,\bn\phi^*)_{\mc{D}} + \(\kappa_e \llb\phi\rrb,
\llb\phi^*\rrb\)_{E_h^i} + \(\llb\phi\rrb,\ldb\mathrm{D}\partial_n
\phi^*\rdb\)_{E_h^i} +\\
& \(\ldb\mathrm{D}\partial_n \phi\rdb,\llb\phi^*\rrb\)_{E_h^i} + \(\kappa_e
\phi,\phi^*\)_{\partial \mc{D}^d} -\frac{1}{2} \(\phi,\mathrm{D} \partial_n
\phi^*\)_{\partial \mc{D}^d} - \frac{1}{2}\(\mathrm{D}\partial_n
\phi,\phi^*\)_{\partial \mc{D}^d}
\label{mip_b}
\end{split}
\end{equation}
\begin{equation}
l(\phi^*) = (Q_0,\phi^*)_{\mc{D}}+ (J^{inc},\phi^*)_{\partial \mc{D}^r}
\label{mip_l}
\end{equation}
where $(f,g)_{\mc{D}} = \sum_{K\in \mathbb{T}_h} \(f,g\)_K$, 
$(f,g)_K = \int_K fg\ d\br$, $(f,g)_{E_h^i}=\sum_{e\in E_h^i}(f,g)_e$, 
$(f,g)_e = \int_e fg\ ds$, $Q_0 = \Sigma_{s,0} \delta \phi^{(l)}$, 
$J^{inc} = \sum_{\bo\cdot\bs{n}_b >0} w_m |\bo_m \cdot \bs{n}_b| \delta
\psi_m^{(l)}$, $\mathbb{T}_h$ is the mesh used to discretize the domain
$\mc{D}$ into nonoverlapping elements $K$, $E_h^i$ is the set of interior
edges, $\mc{D}$  is the spatial domain, $\partial \mc{D}^d$ is the boundary of
the domain with Dirichlet condition, $\partial \mc{D}^r$ is the boundary of
the domain with reflective condition, $\Sigma_a$ is the absorption macroscopic
cross section, D is the diffusion coefficient, $\bs{n}_b$ is the outward
normal unit vector, $\partial_n = \bs{n}_e\cdot \bn$ where $\bs{n}_e$ is the 
normal unit vector associated with a given edge $e$, 
$\llb\phi\rrb = \phi^+ - \phi^-$ is the jump of $\phi$ at the interface between 
two elements, $\ldb\phi\rdb = \frac{\phi^+ + \phi^-}{2}$ is the mean of $\phi$ 
at the interface between two elements, 
$\phi^{\pm}=\lim_{s\rightarrow^{\pm}}\phi(\bs{r}+s\bs{n}_e)$, and
$\kappa_e = \max\(\kappa_e^{IP},\frac{1}{4}\)$
with:
\begin{equation}
\kappa_e^{IP} = \left\{
\begin{aligned}
&\frac{c(p^+)}{2} \frac{\mathrm{D^+}}{h_{\bot}^+} + \frac{c(p^-)}{2}
\frac{\mathrm{D}^-}{h_{\bot}^-} & \textrm{on interior edges, i.e., }
e\in E_h^i\\
&c(p) \frac{\mathrm{D}}{h_{\bot}} & \textrm{on boundary edges, i.e., } e
\in\partial \mc{D}^d 
\end{aligned}
\right. 
\end{equation}
where $c(p)$ is given by $c(p) = 2p (p+1)$, $p$ is the polynomial order ($p=1$
in this paper) and $h_{\bot}$ is the length of the cell in the direction
orthogonal to the edge $e$. On triangular cells, $h_{\bot}=\frac{2A}{L_e}$
with $A$ the area of the triangle and $L_e$ the length of the edge $e$. This
form of MIP was developed for discontinuous finite elements on triangular
cells. Once $\epsilon^{(l+1/2)}=\phi$ is obtained by solving \cref{mip}, the 
scalar flux and, when there is reflective boundaries, the angular fluxes can 
be corrected using:
\begin{align}
\Psi_m^{(l+1)} &= \Psi_m^{(l+1/2)} + \epsilon_m^{(l+1/2)}\\
\Phi^{(l+1)} &= \Phi^{(l+1/2)} + \varepsilon^{(l+1/2)}
\end{align}
By assuming that the angular dependence satisfies a diffusion expansion, the
angular correction can be computed as soon as the scalar flux correction is
known:
\begin{equation}
\epsilon_m^{(l+1/2)} = \frac{1}{4\pi} \(\varepsilon^{(l+1/2)} - 3D \bo_m\cdot \bn 
\varepsilon^{(l+1/2)}\)
\end{equation}

If PWLD finite elements are used, \cref{mip_b,mip_l} become:
\begin{equation}
\begin{split}
b(\phi,\phi^*) =& \la\Sigma_a \phi,\phi^*\ra_{\mc{D}}+
\la\mathrm{D}\bn\phi,\bn\phi^*\ra_{\mc{D}} + \(\kappa_e \llb\phi\rrb,
\llb\phi^*\rrb\)_{E_h^i} + \(\llb\phi\rrb,\ldb\mathrm{D} 
\partial_n\phi^*\rdb\)_{E_h^i} +\\
& \(\ldb\mathrm{D}\partial_n \phi\rdb,\llb\phi^*\rrb\)_{E_h^i} + \(\kappa_e
\phi,\phi^*\)_{\partial \mc{D}^d} -\frac{1}{2} \(\phi,\mathrm{D} \partial_n
\phi^*\)_{\partial \mc{D}^d} - \frac{1}{2}\(\mathrm{D}\partial_n
\phi,\phi^*\)_{\partial \mc{D}^d}
\label{mip_b_pwld}
\end{split}
\end{equation}
\begin{equation}
l(\phi^*) = \la Q_0,\phi^*\ra_{\mc{D}}+ \(J^{inc},\phi^*\)_{\partial \mc{D}^r}
\label{mip_l_pwld}
\end{equation}
where $\la f,g\ra_{\mc{D}} = \sum_{K\in \mathbb{T}_h} \la f,g\ra_{K}$, 
$\la f,g \ra_K = \sum_{s=1}^{V_K} \int_{K_s} fg\ d\br$, $K_s$ is a ``side'' 
sub-cell and $V_K$ is the number of vertices of cell $K$.
When the cells are not triangular, there is no simple way to know
$h_{\bot}$. To simplify the calculation of $h_{\bot}$, we assume that the
polygonal cells are not too far from being regular polygonal cells. If the
cell has an even number of edges, we assume that the orthogonal length is two
times the apothem
$\(\textrm{apothem}=2\times\frac{\textrm{area}}{\textrm{perimeter}}\)$. If the 
cell has an odd number of edges, we assume that the length is the apothem plus the 
circumradius $\(\textrm{circumradius}=\sqrt{\frac{2\times \textrm{area}}{V
\sin\(\frac{2\pi}{V}\)}}\)$. Therefore, $h_{\bot}$ is given by:
\begin{table}[H]
\begin{center}
\begin{tabular}{|c|c|c|c|c|}
\hline
Number of edges & 3 & 4 & $\geq 3$ and odd number & $\geq 4$ and even number\\
\hline
$h_{\bot}$ & $2 \times \frac{\textrm{area}}{L_e}$ &
$\frac{\textrm{area}}{L_e}$ & $4\times
\frac{\textrm{area}}{\textrm{perimeter}}$ & $2 \times
\frac{\textrm{area}}{\textrm{perimeter}}+\sqrt{\frac{2\times
\textrm{area}}{V\sin\(\frac{2\pi}{V}\)}}$\\
\hline
\end{tabular}
\caption{$h_{\bot}$ for different cells.}
\end{center}
\end{table}

\section{Algebraic Multigrid}
\subsection{Introduction}
As mentioned earlier, the most common way to solve a SPD system is to use
conjugate gradient preconditioned with SSOR (PCG-SSOR). In this research, we
will compare the time needed to solve the DSA scheme with PCG-SSOR with the
time needed by CG preconditioned with an algebraic multigrid method as
preconditioner for CG. We will be using the ML package \cite{ml-guide} from
the Trilinos library. ML is a multigrid preconditioning package. This package
uses a smoothed aggregation algebraic multigrid to build a preconditioner for
a Krylov method. We will also use AGMG \cite{agmg_guide}.\\
 
Tire de \cite{amg}\\
Like for geometric multigrid, the algebraic multigrid requires a sequence of
grids, intergrid transfer operators, a smoothing operator, coarse-grid
versions of the fine-grid operator, and a solver for the coarsest grid. In
this section, a grid is defined as a subset of the variables of the problem.
With standard multigrid methods, smooth functions are geometrically or
physically smooth, they have a low spatial frequency. In these cases, we
assume that relaxation smooths the errors and we select a coarse grid that
represents smooth functions accurately. We then choose intergrid operators
that accurately transfer smooth functions between grids. With AMG, we first
select a relaxation scheme that allows us to determine the nature of the
smooth error. Because we do not have access to a physical grid, the sense of
smoothness must be defined algebraically. We define smooth error to be any
error that is not reduced effectively by relaxation.

\cite{k_cycle}\\
Multigrid were initially designed as stand-alone solvers, and they are quite
successful as such in many applications, for which one may refer to the
so-called ``textbook multigrid efficiency'' (meaning that the solutions to the
governing system of equations are attained in a computational work which is a
small (less than 10) multiple of the operation count in the discretized system
of equations (residual evaluations)). However MG methods, especially their
algebraic variants are nowadays used in applications for which such
efficiency is yet to be achieved. One common way to somewhat improve their
robustness is to use them as preconditioners in a Krylov subspace iterative
methods, for instance, in the conjugate gradient method (CG) if the system
matrix is symmetric positive definite (SPD). Now, this still may not be

Tire de \cite{amg_course}\\
Jacobi, Gauss-Seidel and ILU, instead of reducing the error
$e^{(i)}=u-u^{(i)}$, actually only smooth the error. Therefore, after a couple
of smoothing steps, a smooth correction must be added to the approximate
solution. The idea of the multigrid method is to compute the smooth correction
$v_H$ on a coarser grid and interpolate the correction on the fine grid:
\begin{equation}
u^{(i+1)} = u^{(i)}+\bs{P}v_{H}
\end{equation}
The coarse grid correction $v_H$ is the solution of a linear system $\bs{A}_H
v_H = d_H$ on a coarse grid $\bo_H$.

In general, there are two possibilities for the choice of the matrix
$\bs{A}_H$ on the coarse grid. One option is to discretize the partial
differential equation on the coarse grid with the same method which has been
applied on the finest grid. The second possibility:
\begin{equation}
\bs{A}_H = \bs{R}\bs{A}_H \bs{P}
\end{equation}
is called Galerkin approximation. The combination of a smoothing procedure and
a coarse grid correction leads to the two-grid method. The smoothing procedure
$S^{\nu}(u_h,f_h)$ returns an improved solution for the right hand side $f_h$
starting with $u_h$ and computing $\nu$ steps.

Since the exact solution of the coarse grid system $\bs{A}_H v_H = d_H$ in
algorithm (\ref{2_1_1}) is usually very time consuming, it is recursively
replaced by $\gamma$ two-grid iteration steps. This yields the multigrid
algorithm:
\begin{algorithm}
Let a hierarchy of grids $\bo_0 \subset \bo_1 \subset \hdots \subset
\bo_{l_{\max}}$, prolongation and restriction operators $P_{l,l-1}$,
$R_{l-1,l}$ between these grids, matrices $\bs{A}_l$ and smoothing iterations
$\bs{S}_l$ on these grids be given. Then, the algorithm
$MG(u_{l_{\max}},f_{l_{\max}},l_{l_{\max}})$ defines the approximate inverse
$M_{MG}^{-1}$ on the finest grid $\bo_{l_{\max}}$.
\begin{itemize}
\item[] if $(l=0)$
\begin{itemize}
\item[] $u_l = \bs{A}_l^{-1} f_l$
\end{itemize}
\item[] else
\item[] \{
\begin{itemize}
\item[] $u_l=S_l^{\nu_1}(u_l,f_l)$
\item[] $d_{l-1} = R_{l-1,l}(f_l-\bs{A}_l u_l)$
\item[] $v_{l-1} = 0$
\item[] for $(j=0; j<\gamma;++j)$
\begin{itemize}
\item[] $MG (v_{l-1},d_{l-1},l-1)$
\end{itemize}
\item[] $u_l=u_l+P_{l,l-1} v_{l-1}$
\item[] $u_l=S_l^{\nu_2}(u_l,f_l)$
\end{itemize}
\item[] \}
\end{itemize}
\label{2_1_1}
\end{algorithm}
For $\gamma = 1$ or $\gamma = 2$, the method is called $V(\nu_1,\nu_2)-$cycle
or $W(\nu_1,\nu_2)-$cycle respectively. 
\subsection{Aggregation methods}
In interpolation methods, typically, each coarse grid degree of freedom has a
directly associated degree of freedom on the fine grid. Since aggregation
methods cluster on the fine grid unknowns to aggregates representing the
unknowns on the coarse grid, aggregation methods do not allow such a simple
identification of degrees of freedom on the coarse and the fine grid.

Pris de \cite{ml-guide}\\
\subsection{Smoothed Aggregation Options}
A graph of the matrix is usually constructed by associating a vertex with each
equation and adding an edge between two vertices $i$ and $j$ if there is a
nonzero in the $(i,j)^{th}$ or $(j,i)^{th}$ entry. Is is this matrix graph
whose vertices are aggregated together that effectively determines the next
coarser mesh. Uncoupled aggregation and MIS aggregation schemes use fixed
ratio of coarsening between levels. Poorly done aggregation can adversely
affect the multigrid convergence and the time per iteration. In particular, if
the scheme coarsens too rapidly multigrid convergence may suffer. However, if
coarsening is too slow, the number of multigrid increases and the number of
nonzeros per row in the coarse grid discretization matrix my grow rapidly. 

\subsection{Tire de \cite{mis} = Maximally Independent Sets}
The $\bs{P}_k$ (interpolation operators that transfer solutions from coarse
grids to finer grids) are the key ingredients that must be determined
automatically within an algebraic multigrid methods. The smooth aggregation
$\bs{P}_k$ are determined in two steps: coarsening and grid transfer
construction. The first step is to take each grid point and assign it to an
aggregate. The second step consists of first forming a tentative prolongator
matrix $\tilde{\bs{P}}_k$ and then applying a prolongator smoother
$\tilde{\bs{S}}_k$ to it giving rise to $\bs{P}_k =
\tilde{\bs{S}}_k\tilde{\bs{P}}_k$. The tentative prolongator matrix
$\tilde{\bs{P}}_k$ are as follows (for specific applications such as
elasticity problems, more complicated tentative prolongators can be derived
based on rigid body motions):
\begin{equation}
\tilde{\bs{P}}_k(i,j) = \left\{
\begin{aligned}
&1 &\textrm{if }i^{th}\textrm{ point is contained in }j^{th}\textrm{
  aggregate}\\
& 0 &\textrm{otherwise}
\end{aligned}
\right.
\end{equation}
The tentative prolongator can be viewed as a simple grid transfer operator
corresponding to piecewise constant interpolation. While $\tilde{\bs{P}}_k$
can be used for $\bs{P}_k$, a more robust method is realized by smoothing the
piecewise constant basis functions. For example, applying a damped Jacobi
smoother yields:
\begin{equation}
\begin{split}
\bs{P}_k &= (\bs{I}-\omega \bs{D}_k^{-1} \bs{A}_k) \tilde{\bs{P}}_k \\
&= \bs{S}_k \tilde{\bs{P}}_k
\end{split}
\label{P_k}
\end{equation}
If $\omega$ and the aggregates are properly chosen, \cref{P_k} leads to linear
interpolation when applied to an one-dimensional Poisson problem. In general,
\cref{P_k} does not correspond to linear interpolation but yields better
interpolation than piecewise constant interpolation. Small aggregates lead to
high iteration costs. This is because the number of unknowns on the next fines
grid is equal to the number of aggregates and because the number of nonzeros
per row in the coarse grid discrete operator depends on the distance between
non-neighboring aggregates. However, aggregates that are too large lead to
grid transfer operators that look more like piecewise constant interpolation
and give poorer convergence rate. The basic aggregation procedure (like the
one used in Uncoupled) is given below:
\begin{description}
\item[Phase 1:] repeat until all unaggregated points are adjacent to an
aggregate:
\begin{enumerate}
\item pick root point not adjacent to any existing aggregate
\item define new aggregate as root point plus all its neighbors
\end{enumerate}
\item[Phase 2:] sweep unaggregated points into existing aggregates or use them
to form new aggregates.
\end{description}

Pris de \cite{smooth_agg_conv}\\
The hierarchy of coarse level matrices is defined by:       
\begin{align}
&\bs{A}_{l+1} = \tilde{\bs{P}}_l^T \bs{A}_l \tilde{\bs{P}}_l\\
&\bs{A}_1 = \bs{A}
\end{align}
Although we will carry out some convergence estimates for general prolongators
smoothers $\bs{S}_l ; \mathbb{R}^{n_l}\rightarrow \mathbb{R}^{n_l}$, the form
of $\bs{S}_l$, we use is:
\begin{equation}
\bs{S}_l = \bs{I} - \frac{4}{3\bar{\lambda}_l^M} \bs{M}_l^{-1} \bs{A}_l
\end{equation}
Here, $\bar{\lambda}_l^M > \(\bs{M}_l^{-1} \bs{A}_l\)$, $\rho$ denotes the
spectral radius, and:
\begin{align}
&\bs{M}_l = \(\bs{P}_l^1\)^T \bs{P}_l^1\\
&\bs{P}_l^1 = \bs{P}_2^1\hdots\bs{P}_l^{l-1}\\
&\bs{P}_1^1 = \bs{I}
\end{align}
The mapping $\bs{P}_l^1 : \mathbb{R}^{n_l}\rightarrow \mathbb{R}^{n_1}$ is
called composite tentative prolongator. The parameter 4/3 on level $l$
minimizes the value of $\rho(\bs{M}_{l+1}^{-1}\bs{A}_{l+1})$.


Pris de \cite{amg_unstruc}\\
\subsection{AGMG}
\cite{agmg_guide}\\
Pris de \cite{k_cycle}\\
Now, this still may not be sufficient to provide fast convergence if the 
two-grid convergence factor is too large to allow convergence properties 
independent of the number of levels with $V-$ or $W-$cycles. Moreover, in 
real-life problems, it is often impossible to predict if such a situation 
will occur or not, and the type of cycle that would be optimal. This motivates 
us to consider Krylov-based MG-cycles (or $K-$cycle, for short). With these 
cycles, the MG method is still based on the recursive use of a two-grid method, 
but the needed coarse-grid solve is defined by a few steps of a Krylov subspace 
iterative method with the already defined (by recursion) MG method on the 
previous (coarser) level as preconditioner. If $\mu$ inner iterations are 
performed at each level, we have more specifically a $K_{\mu}-$cycle 
preconditioner. Such an idea is not new; it has been used, also in a multilevel 
setting, for the so-called AMLI methods. The latter can be viewed as stabilized 
versions of the hierarchical basis methods. The stabilization comes from more 
than one recursive calls of the preconditioner defined (by recursion) at a given 
level. Observe that the MG preconditioner defined in this way becomes a 
nonlinear operator and thus, the analysis of such techniques is not straightforward. 
Here, we consider the flexible CG method \cite{fcg,fcg_2,fcg_3,fcg_4}. For 
difficult problems, for which $V-$cycle MG is slow, $K-$cycle MG can be much 
more effective than $W-$cycle MG. $K-$cycle MG appears more robust than $W-$cycle 
MG. It can exhibit convergence properties independent of the number of levels 
even when the condition number for the underlying two-grid method is relatively 
large. Using $K-$cycles may thus enhance the robustness of a MG method, in 
particular that of AMG schemes for real-life problems. This enhanced robustness is
obtained nearly for free since the $K-$cycle has roughly the same
computational complexity as the $W-$cycle. Finally, sometimes the number of
unknowns does not decrease sufficiently fast from one level to the next to
allow inner iterations at each level as foreseen with standard $K-$ or
$W-$cycles. To cope with such cases, we introduce a variant of $K-$cycle MG
that allows inner iterations only at levels of given multiplicity $k_0>1$,
whereas a $V-$cycle formulation is used at other levels.

\cite{agmg2}\\
We consider more particularly aggregation-based multigrid schemes, in which
the hierarchy of coarse systems is obtained from a mere aggregation of the
unknowns. This approach is obtained from a mere aggregation of the unknowns.
This approach is sometimes referred to as ``plain'' or ``unsmoothed''
aggregation, to distinguish it from ``smoothed aggregation AMG''. It has some
appealing features such as cheap setup stage and memory requirements. However,
it is somehow nonstandard because it does not mimic any well-established
geometric multigrid method. We show that the
convergence rate can be bounded assessing for each aggregate a local quantity
which in some sense measure its quality, the bound being determined by the
worst aggregate's quality. Automatic aggregation algorithms tend to always
produce a limited number of badly shaped aggregates, and since, the bound is
determined by the worst aggregate's quality, even a few of these can
significantly impact the resulting bound. Here we overcome these limitations
mainly by introducing a new aggregation algorithm based on the explicit
control of aggregate's quality. It tends to optimize the latter while imposing
some minimal requirements; i.e., the algorithm has as main input parameter the 
upper bound on the two-grid condition number that is required to hold. Of course, 
such an approach potentially induces an increase of the algorithmic complexity. 
In the multigrid context, mastering the complexity means ensuring that the cost of
each iteration step does not grow more than linearly with the number of the
unknowns. Taking into account that the two-grid method has to be used
recursively, this further means ensuring that the number of unknowns decreases
sufficiently fast from one level to the next. With aggregation-based methods,
the factor by which the number of unknowns is decreased is actually the mean
aggregates' size (i.e., the mean number of unknowns inside an aggregate).
Whereas with heuristic aggregation algorithms it is relatively easy to control
this mean aggregates' size, the present approach introduces more uncertainty:
one may form aggregates of a given target size, but there is no a priori
guarantee that they will satisfy the quality criterion that allows to control
the condition number. All in all, there are two advantages in controlling
explicitly the convergence rate (or the condition number) instead of the
complexity. Firstly, in a typical situation, a ``complexity oriented''
algorithm will form a few badly shaped aggregates. As already mentioned, this
may have a dramatic impact on the convergence analysis. If there are only a
few such bad aggregates, their influence on the actual convergence may or not
be significant, in general we just don't know. An algorithm that explicitly
controls the condition number will refuse to form these bad aggregates and
stay instead with a few aggregates of smaller size or even some unaggregated
nodes. Compared with the previous situation, here we know that if there is
only a small amount of such aggregates, then the impact on the efficiency of
the method will be minor, since it would only affect the mean aggregates' size
in a unessential way.

The aggregation scheme considered in this work is based on a few passes of a
pairwise matching algorithm, which amounts to group into pairs. The qualities
$\mu(G)$ is a inferior bound of the two-grid condition numbers.


Tire de \cite{amg_pn}\\
This work describes the application of the algebraic multigrid method to the
solution of the even-parity finite element-spherical harmonics (FE-$P_N$)
method. The AMG preconditioner led to a 60\% reduction in the solution time
compare to ILU(0) and even more compare to SSOR.

%%%%%%%%%%%%%%%%%%%%%%%%%%%%%%%%%%%%%%%%%%%%%%%%%%%%%%%%%%%%%%%%%%%%%%%%%%%%%%%%%%
%%%%%%%%%%%%%%%%%%%%%%%%%%%%%%%%%%%%%%%%%%%%%%%%%%%%%%%%%%%%%%%%%%%%%%%%%%%%%%%%%%
\section{Results} \label{sec_res}
%%%%%%%%%%%%%%%%%%%%%%%%%%%%%%%%%%%%%%%%%%%%%%%%%%%%%%%%%%%%%%%%%%%%%%%%%%%%%%%%%%
%%%%%%%%%%%%%%%%%%%%%%%%%%%%%%%%%%%%%%%%%%%%%%%%%%%%%%%%%%%%%%%%%%%%%%%%%%%%%%%%%%

In this section, we present two Fourier analyses: one where the $S_n$ order is
varied and one where the cell aspect ratio is varied and compare different
methods to solve MIP: Conjugate Gradient (CG), Conjugate Gradient
Preconditioned with Symmetric Gauss-Seidel (PCG-SGS), Conjugate Gradient
Preconditioned with ML using Uncoupled aggregation (PCG-MLU),
Conjugate Gradient Preconditioned with ML using MIS aggregation (PCG-MLMIS),
and AGMG. 

%%%%%%%%%%%%%%%%%%%%%%%%%%%%%%%%%%%%%%%%%%%%%%%%%%%%%%%%%%%%%%%%%%%%%%%%%%%%%%%%%%
\subsection{Fourier Analyses}
%%%%%%%%%%%%%%%%%%%%%%%%%%%%%%%%%%%%%%%%%%%%%%%%%%%%%%%%%%%%%%%%%%%%%%%%%%%%%%%%%%

A Fourier analysis is often performed to assess some of the properties of 
DSA-accelerated transport solves \cite{larsen_dsa,consistent_p1}. In a Fourier analysis,
the error in the solution is decomposed into Fourier modes. By inspecting the 
damping of the different error modes, the effectiveness of a SI+DSA scheme can 
be studied, often in an infinite homogeneous problem. The largest damping
factor is the spectral radius of the method. The smaller the spectral radius,
the faster the scheme will converge. \textcolor{red}{A spectral radius greater than one denotes
an unstable method.} 

\subsubsection{Spectral radius as a function of the \sn order}
%%%%%%%%%%%%%%%%%%%%%%%%%%%%%%%%%%%%%%%%%%%%%%%%%%%%%%%%%%%%%%%%%%%%%%%%%%%%%%%%%%

This Fourier analysis is carried on a square cell, using a
Gauss-Legendre-Chebyshev (GLC) angular quadrature. The medium is homogeneous with a 
scattering ratio $c=0.9999$; periodic boundary conditions are used. The results 
are plotted on \Cref{fig_fa_sn}, where the $x-$axis is the mesh size in mean free 
paths and the $y-$axis is the spectral radius. There are four curves corresponding 
to different $S_n$ orders: $S_2$, $S_4$, $S_8$ and $S_{16}$.
\begin{figure}[H]
  \centering
  \includegraphics[width=0.5\textwidth]{sn_order_9999}
  \caption{Fourier analysis as a function of the mesh optical thickness, square cell,
    various \sn orders.}
  \label{fig_fa_sn}
\end{figure}
From \Cref{fig_fa_sn}, we can conclude that MIP is stable for every 
cell size. The spectral radius is always less than 0.5, except for $S_2$ where 
it is about 0.7. In the fine mesh limit, the spatial continuum results are recovered:
the spectral radius of SI+DSA using an $S_2$ quadrature in 2D is $0.5c$; as the angular 
quadrature is refined, the standard result of $0.2247c$ for the spectral result is obtained.

\subsubsection{Spectral radius as a function of the cell aspect ratio}
%%%%%%%%%%%%%%%%%%%%%%%%%%%%%%%%%%%%%%%%%%%%%%%%%%%%%%%%%%%%%%%%%%%%%%%%%%%%%%%%%%
For this Fourier analysis, we use a $S_{16}$ GLC quadrature. The medium is
again homogeneous with $c=0.9999$ and periodic boundary conditions apply. 
On \Cref{fig_fa_ar}, the five curves correspond to the following cell aspect 
ratios: $\frac{Y}{X}=\frac{1}{16}$, $\frac{Y}{X}=\frac{1}{4}$,
$\frac{Y}{X}=1$, $\frac{Y}{X}=4$, $\frac{Y}{X}=16$, and $\frac{Y}{X}=100$.
\begin{figure}[H]
  \centering
  \includegraphics[width=0.5\textwidth]{aspect_ratio_9999_2}
  \caption{Fourier analysis as a function of the mesh optical thickness,
  $S_{16}$, various aspect ratios.}
  \label{fig_fa_ar}
\end{figure}
MIP is stable for every aspect ratio, including 100, and the maximum spectral radius
is about 0.5.

%%%%%%%%%%%%%%%%%%%%%%%%%%%%%%%%%%%%%%%%%%%%%%%%%%%%%%%%%%%%%%%%%%%%%%%%%%%%%%%%%%
\subsection{Numerical results using MIP-DSA implemented in an $S_n$ code}
%%%%%%%%%%%%%%%%%%%%%%%%%%%%%%%%%%%%%%%%%%%%%%%%%%%%%%%%%%%%%%%%%%%%%%%%%%%%%%%%%%

\subsubsection{Homogeneous test problems}  \label{sec_homog}
%%%%%%%%%%%%%%%%%%%%%%%%%%%%%%%%%%%%%%%%%%%%%%%%%%%%%%%%%%%%%%%%%%%%%%%%%%%%%%%%%%

We compare different solvers for MIP-DSA using a homogeneous medium, $100cm
\times 100cm$, $\Sigma_t = 1cm^{-1}$ and $\Sigma_s = 0.999cm^{-1}$, with
vacuum boundary conditions and a source of intensity $1cm^{-3}s^{-1}$. We
use a $S_8$ GLC angular quadrature, Source Iteration as solver
with relative tolerance of $10^{-8}$ and a relative tolerance of
$10^{-10}$ for MIP-DSA. The medium is discretized using two different meshes:
\begin{enumerate}
  \item[Quadrilateral grid:] the mesh is composed of 49236 quadrilateral
    cells that to 197052 degrees of freedom.
  \item[Polygonal grid:] the mesh is composed of 45204 triangles, 823
    quadrilaterals, 4978 pentagons, 4155 hexagons, 725 heptagons, and 24
    octagons, for a total of 55909 cells and 193991 degrees of freedom. This
    example will allow us to test MIP and the different preconditioners on a
    mesh composed of different cell types.
\end{enumerate}
%
The meshes and the numerical solutions are given on \Cref{homog_test}.
In \Cref{comparison_homog_quad}, the different solvers, used with the
quadrilateral grid, are compared.
%
\begin{figure}[H]
  \centering
  \begin{subfigure}{0.45\textwidth}
    \centering
    \includegraphics[width=\textwidth]{big_homog_quad_crop}
    \caption{Quadrilateral cells}
  \end{subfigure}
  \begin{subfigure}{0.45\textwidth}
    \centering
    \includegraphics[width=\textwidth]{big_homog_poly_crop}
    \caption{Polygonal cells}
  \end{subfigure}
  \caption{Meshes and scalar fluxes}
  \label{homog_test}
\end{figure}
%
\begin{table}[H]
  \begin{center}
    \caption{Comparison of different preconditioners for quadrilateral cells}
    \begin{tabular}{|c|c|c|c|c|c|c|}
      \hline
      & No-DSA & CG & PCG-SGS & PCG-MLU & PCG-MLMIS & AGMG \\
      \hline
      SI iterations   & 7311    & 24      & 24       & 24      & 24      & 24 \\
   Precond init (s)   & NA      & NA      & 0.171358 & 1.8255  & 9.56078 & 0.332 \\
MIP calculation (s)   & NA      & 1095.7  & 1311.76  & 192.622 & 197.632 & 29.9727 \\
      CG iterations   & NA      & 56649   & 17332    & 630     & 604     & 578 \\
Total calculation (s) & 39176.7 & 1264.98 & 1477.95  & 363.202 & 367.841 &
      194.568 \\
      \hline
    \end{tabular}
    \label{comparison_homog_quad}
  \end{center}
\end{table}
In \Cref{comparison_homog_quad}, SI iterations is the number iteration of 
Source Iteration needed to solve the problem, Precond init is the time, in
seconds, needed to initialize the preconditioner used by CG, MIP calculation
is the total time, in seconds, spent solving DSA during the calculation, CG
iterations is the total number of CG iterations used to solve MIP, and Total
calculation is the time, in seconds, needed to solve the problem. We can see
than PGC-ML and AGMG require about the same number of iterations (two orders
of magnitude less than for CG). However, AMG is much faster than PCG-ML. This 
is because each PCG-SGS iteration is much slower than one
unpreconditioned CG iterations. Profiling of the code reveals that the
bottleneck is the function 
\emph{Ifpack\_PointRelaxation::ApplyInverseSGS\_FastCrsMatrix} of Trilinos. This
function applies the forward and the backward substitutions required by SGS.
It is unclear why these substitutions are so costly. We can see in
\Cref{comparison_homog_quad}, the reason why there is such a big
difference between AGMG and PCG-ML is due to SGS. SGS is used as pre- and
post-smoother in ML and the function
\emph{Ifpack\_PointRelaxation::ApplyInverseSGS\_FastCrsMatrix} is once again the
bottleneck of the method.

The different solvers, used with the polygonal grids, are compared in
\Cref{comparison_homog_poly}:
\begin{table}[H]
  \begin{center}
    \caption{Comparison of different preconditioners for polygonal cells}
    \begin{tabular}{|c|c|c|c|c|c|c|}
      \hline
      & No-DSA & CG & PCG-SGS & PCG-MLU & PCG-MLMIS & AGMG \\
      \hline
      SI iterations   & 7311    & 23      & 23      & 23      & 23      & 23 \\
   Precond init (s)   & NA      & NA      & 0.06388 & 1.73379 & 8.0426  & 0.388 \\
MIP calculation (s)   & NA      & 877.861 & 1263.31 & 198.63  & 191.989 &
      31.242 \\
      CG iterations   & NA      & 46262   & 16712   & 652     & 603     & 555 \\
Total calculation (s) & 42666.7 & 1060.53 & 1447.53 & 382.275 & 384.422 &
      216.946 \\
      \hline
    \end{tabular}
    \label{comparison_homog_poly}
  \end{center}
\end{table}
We see that using different types of cells in the same mesh does not affect
the performance of MIP-DSA or that of its preconditioners.

\subsubsection{Heterogeneous medium}
%%%%%%%%%%%%%%%%%%%%%%%%%%%%%%%%%%%%%%%%%%%%%%%%%%%%%%%%%%%%%%%%%%%%%%%%%%%%%%%%%%

In this example, a heterogeneous geometry with three materials is used. It is 
composed of 184 triangles, 3720 quadrilaterals and 2791 regular hexagons of 
side $0.05cm$ for a total of 6695 cells and 32178 degrees of freedom (spatial 
unknowns per angular direction). The domain is $5.28275cm$ by $4.6cm$. 
Reflective boundary conditions are used. A $S_{16}$ GLC 
quadrature is used. The SI solver has a relative tolerance of 
$10^{-8}$ and the relative tolerance for MIP is $10^{-10}$. \Cref{hex_zones}
shows the problem geometry and the material properties are in given
\Cref{hex_prop}.
\begin{figure}[H]
  \centering
  \includegraphics[width=0.4\textwidth]{source_crop}
  \caption{Zones of the heterogeneous test domain}
  \label{hex_zones}
\end{figure}
\begin{table}[H]
  \begin{center}
    \caption{Properties of the different zones}
    \begin{tabular}{|c|c|c|c|}
      \hline
       & Inner region & Intermediate region & Outer region \\
      $\Sigma_t$ $(cm^{-1})$ & 1.5 & 1.0 & 1.0 \\
      $\Sigma_s$ $(cm^{-1})$ & 1.4999 & 0.999 & 0.3 \\
     source $(cm^{-3}s^{-1}$ & 1.0 & 0.0 & 0.0 \\
      \hline
    \end{tabular}
    \label{hex_prop}
  \end{center}
\end{table}
The different solvers are compared in \Cref{comparison_hex}.
\textcolor{red}{really think problem, why?}
\begin{table}[H]
  \begin{center}
    \caption{Comparison of preconditioners, heterogeneous problem}
    \begin{tabular}{|c|c|c|c|c|c|c|}
      \hline
      & No-DSA & CG & PCG-SGS & PCG-MLU & PCG-MLMIS & AGMG\\
      \hline
      SI iterations & 278     & 17      & 17        & 17       & 17      & 17  \\
   Precond init (s) & NA      & NA      & 0.0160661 & 0.368768 & 1.41632 &
      0.07  \\
MIP calculation (s) & NA      & 58.422  & 126.93    & 33.2225  & 31.3045 &
      2.924 \\
      CG iterations & NA      & 12214   & 6679      & 415      & 386     & 248  \\
Total calculation (s) & 910.566 & 120.889 & 190.413 & 99.7524  & 97.4666 &
      70.6424 \\      
      \hline
    \end{tabular}
    \label{comparison_hex}
  \end{center}
\end{table}
We can see that the remarks of \Cref{sec_homog} made for the homogeneous test
remain the same. MIP-DSA is effective even with heterogeneous medium and AGMG is
still the fastest solver. It is interesting to note that, contrary to the
homogeneous tests where the number of CG iterations remained similar for all
algebraic multigrid preconditioners, in this heterogeneous test AGMG requires
significantly fewer iterations than the Trilinos implementation, PCG-MLU, 
and PCG-MLMIS.

\subsubsection{Locally refined grid}
%%%%%%%%%%%%%%%%%%%%%%%%%%%%%%%%%%%%%%%%%%%%%%%%%%%%%%%%%%%%%%%%%%%%%%%%%%%%%%%%%%

In this example \cite{mip}, the domain is $10cm\times 10cm$. The left and bottom
boundaries are reflective whereas the right and the top boundaries are vacuum. 
There are 10720 cells: 10482 quadrilaterals, 236 pentagons,
and 2 hexagons for a total of 43120 degrees of freedom. 
As in the previous example, the domain is composed if three zones (see
\Cref{zone_amr} and \Cref{prop_amr}).
\begin{figure}[H]
  \centering
  \includegraphics[width=6cm]{zone_amr}
  \caption{Zones of the AMR test domain}
  \label{zone_amr}
\end{figure}
\begin{table}
  \begin{center}
    \caption{Material properties, AMR-like test problem}
    \begin{tabular}{|c|c|c|c|}
      \hline
      & Inner region & Intermediate region & Outer region  \\ \hline
    $\Sigma_t$ $(cm^{-1})$ & 1.5  & 1.  & \textcolor{red}{1.0 ?} \\
    $\Sigma_s$ $(cm^{-1})$ & 1.44 & 0.9 & 0.3 \\
  Source $(cm^{-3}s^{-1})$ & 1.0  & 0.0 & 0.0 \\
      \hline
    \end{tabular}
    \label{prop_amr}
  \end{center}
\end{table}
The distribution of cells is given \Cref{fig_pol_dist}.
\begin{figure}[H]
  \centering
  \includegraphics[width=6cm]{polygon_amr}
  \caption{Polygons distribution}
  \label{fig_pol_dist}
\end{figure}
where the blue cells are quadrilaterals, the green cells are pentagons, and
the red cells are hexagons. This mesh is typical of a mesh obtained after one 
level of adaptive mesh
refinement (the cells at the interface of different materials have been refined
once). We see that instead of introducing hanging nodes, we have introduce
pentagons and hexagons in the mesh.
A $S_{16}$ GLC quadrature is employed. The tolerance on SI is $10^{-8}$ and
the tolerance on the CG solvers is $10^{-10}$.
The different solvers are compared in \Cref{table_amr}.
\begin{table}[H]
  \caption{Comparison of preconditioners on AMR mesh}
  \begin{center}
    \begin{tabular}{|c|c|c|c|c|c|c|}
      \hline
       & No-DSA & CG & PCG-SGS & PCG-MLU & PCG-MLMIS & AGMG \\
      \hline
   SI iterations & 184     & 19      & 19       & 19      & 19       & 19 \\
Precond init (s) & NA      & NA      & 0.043463 & 0.358002 & 1.19301 & 0.0111\\
MIP calculation (s) & NA   & 48.1908 & 81.0992  & 25.2699 & 25.0699  & 
      2.56198\\
   CG iterations & NA      & 11300   & 4734     & 361     & 361      & 264 \\
     Total calculation (s) & 802.985 & 138.825 & 172.423  & 116.018 & 116.517  &
      94.1963\\
      \hline
    \end{tabular}
    \label{table_amr}
  \end{center}
\end{table}
The conclusions are similar to the ones made for our previous tests.

\subsubsection{High aspect ratio grid}
%%%%%%%%%%%%%%%%%%%%%%%%%%%%%%%%%%%%%%%%%%%%%%%%%%%%%%%%%%%%%%%%%%%%%%%%%%%%%%%%%%

\red{When the aspect ratio is high and the scattering ratio is close to one, MIP
becomes ill-conditioned.} In the next two examples, the domain is square $100cm
\times 100cm$ with vacuum boundaries. There are 10000 cells and thus, 40000
degrees of freedom. The relative tolerance on SI is $10^{-8}$ and the relative
tolerance for CG is $10^{-10}$. We use a $S_8$ GLC quadrature, $\Sigma_t =
1cm^{-1}$, and $\Sigma_s = 0.999cm^{-1}$. The source is $1n/(cm^3s)$. In the
first test, the domain is discretized by 100 cells along $x$ and 100 cells
along $y$, i.e., square cells with aspect ratio of one. In the second run, 
the domain is discretized by 1000 cells along $x$ and 10 cells along $y$ 
(the aspect ratio is 100).
\begin{table}[H]
  \caption{Comparison of preconditioners on rectangular grid with an aspect
  ratio of 1}
  \begin{center}
    \begin{tabular}{|c|c|c|c|c|c|c|}
      \hline
       & No-DSA & CG & PCG-SGS & PCG-MLU & PCG-MLMIS & AGMG \\
      \hline
      SI iterations & 7311      & 21      & 21      & 21       & 21      & 21 \\
   Precond init (s) & NA        & NA      & 0.01422 & 0.051373 & 1.13144 &
      0.044 \\
MIP calculation (s) & NA        & 32.3825 & 73.8422 & 24.0707  & 25.0065 &
      1.7114 \\
      CG iterations & NA        & 8363    & 4853    & 376      & 375     &
      221\\
Total calculation (s) & 7356.96 & 56.8993 & 98.2609 & 50.1247  & 51.5396 &
      25.9306 \\
      \hline
    \end{tabular}
    \label{table_ar_1}
  \end{center}
\end{table}
\begin{table}[H]
  \caption{Comparison of preconditioners on rectangular grid with an aspect
  ratio of 100}
  \begin{center}
    \begin{tabular}{|c|c|c|c|c|c|c|}
      \hline
       & No-DSA & CG & PCG-SGS & PCG-MLU & PCG-MLMIS & AGMG \\
      \hline
      SI iterations & 7304    & 24      & 24        & 24       & 24      & 24 \\
   Precond init (s) & NA      & NA      & 0.0164239 & 0.362463 & 1.03128 & 0.052 \\
MIP calculation (s) & NA      & 372.227 & 742.902   & 941.06   & 922.258 &
      6.93176 \\
      CG iterations & NA      & 84802   & 43466     & 14180    & 13896   & 821 \\
Total calculation (s) & 9035.6 & 414.301 & 784.77   & 985.796  & 966.77  &
      44.7032 \\
      \hline
    \end{tabular}
    \label{table_ar_100}
  \end{center}
\end{table}                  
As expected, solving the MIP-DSA equations requires more CG iterations when the aspect
ratio increases. PCG-MLU and PCG-MLMIS are significantly more affected by the
increase in the aspect ratio than the other methods. AGMG is the least
affected by the change of aspect ratio and is the best performing
preconditioner.

%%%%%%%%%%%%%%%%%%%%%%%%%%%%%%%%%%%%%%%%%%%%%%%%%%%%%%%%%%%%%%%%%%%%%%%%%%%%%%%%%%
%%%%%%%%%%%%%%%%%%%%%%%%%%%%%%%%%%%%%%%%%%%%%%%%%%%%%%%%%%%%%%%%%%%%%%%%%%%%%%%%%%
\section{Conclusions} \label{sec_conc}
%%%%%%%%%%%%%%%%%%%%%%%%%%%%%%%%%%%%%%%%%%%%%%%%%%%%%%%%%%%%%%%%%%%%%%%%%%%%%%%%%%
%%%%%%%%%%%%%%%%%%%%%%%%%%%%%%%%%%%%%%%%%%%%%%%%%%%%%%%%%%%%%%%%%%%%%%%%%%%%%%%%%%
We have extended the Modified Interior Penalty (MIP) form of Diffusion Synthetic Acceleration (DSA) 
to arbitrary polygonal grids discretized with Piece-Wise Linear Discontinuous finite elements. 
The MIP-DSA solves employ the same discontinuous finite element trial spaces of the discrete \sn
transport equation.
%We proposed a simple way to compute the penalty coefficient on such grids. 
%
Numerical tests have been run on different grids with various types of polygonal cells,
demonstrating the effectiveness of MIP as a diffusion synthetic accelerator for 
\sn transport on arbitrary polygonal grids.
%
%. The advantage of is the potential 
%reduction of the numbers of unknowns and the possibility to use adaptive mesh 
%refinement without having hanging nodes. 
%
Fourier analyses show that a PWLD discretization of MIP diffusion synthetic accelerator 
is always stable, including for high-aspect ratio cells. 
%
Numerical experiments have been performed with mesh containing various types of polygonal cells, including
grids with different polygon types for a given problem and tests with degenerate polygons that mimic
grids obtained in adaptive mesh refinement strategies. In these tests, MIP-DSA always performed effectively,
reducing significantly the number of Source Iterations needed for convergence. 
We noted that the efficiency of MIP does not seem to \textcolor{red}{be affected by the meshes employed}. 
%
The MIP-DSA  matrix is SPD and we solved the corresponding linear system of equations using a standard Conjugate
Gradient method with different preconditioners. 
Algebraic multigrid techniques were found to be the most effective preconditioner with AGMG being more than 20 times
faster than unpreconditioned CG.



% bibliography
\bibliographystyle{unsrt}
\bibliography{biblio}

\end{document}

