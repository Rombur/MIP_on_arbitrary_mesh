%%%%%%%%%%%%%%%%%%%%%%%%%%%%%%%%%%%%%%%%%%%%%%%%%%%%%%%%%%%%%%%%%%%%%%%%%%%%%%%%%%
%%%%%%%%%%%%%%%%%%%%%%%%%%%%%%%%%%%%%%%%%%%%%%%%%%%%%%%%%%%%%%%%%%%%%%%%%%%%%%%%%%
\section{Conclusions} \label{sec_conc}
%%%%%%%%%%%%%%%%%%%%%%%%%%%%%%%%%%%%%%%%%%%%%%%%%%%%%%%%%%%%%%%%%%%%%%%%%%%%%%%%%%
%%%%%%%%%%%%%%%%%%%%%%%%%%%%%%%%%%%%%%%%%%%%%%%%%%%%%%%%%%%%%%%%%%%%%%%%%%%%%%%%%%

We have extended the Modified Interior Penalty (MIP) form of Diffusion Synthetic Acceleration (DSA) 
to arbitrary polygonal grids discretized with Piece-Wise Linear Discontinuous finite elements. 
%We proposed a simple way to compute the penalty coefficient on such grids. 
Tests were run on meshes with various types of polygonal cells,
%. The advantage of is the potential 
%reduction of the numbers of unknowns and the possibility to use adaptive mesh 
%refinement without having hanging nodes. 
Fourier analyses (with isotropic scattering) on rectangular cells show
that this PWLD discretization MIP-DSA is always stable, including for high-aspect ratio cells. 
Numerical experiments have been performed on several polygonal grids, including
tests with different polygon types for a given grid and tests with degenerated polygons that mimic
grids obtained in adaptive mesh refinement strategies. MIP-DSA always performed satisfactorily,
reducing significantly the number of  Source Iterations needed for convergence. 
We noted that the efficiency of MIP does not seem to be affected by such these meshes. 
The MIP-DSA matrix is SPD and we compared different preconditioners for CG to solve the resulting equations. 
Algebraic multigrid methods were found to be the best preconditioner, with AGMG being more than 20 times
faster than CG without preconditioning.

