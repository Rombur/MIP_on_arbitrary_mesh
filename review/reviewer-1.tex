\documentclass{article}
%%%%%%%%%%%%%%%%%%%%%%%%%%%%%%%%%%%%%%%%%%%%%%%%%%%%%%%%%%%%%%%%%%%%%%%%%%%%%%%%%%%%%%%%%%%%%%%%%%%%%%%%%%%%%%%%%%%%%%%%%%%%%%%%%%%%%%%%%%%%%%%%%%%%%%%%%%%%%%%%%%%%%%%%%%%%%%%%%%%%%%%%%%%%%%%%%%%%%%%%%%%%%%%%%%%%%%%%%%%%%%%%%%%%%%%%%%%%%%%%%%%%%%%%%%%%
\usepackage{amsmath,amssymb}
% more math
\usepackage{amsfonts}
\usepackage{amssymb}
\usepackage{amstext}
\usepackage{amsbsy}

\usepackage{color}
\newcommand{\mt}[1]{\marginpar{\small #1}}
%%%%%%%%%%%%%%%%%%%%%%%%%%%%%%%%%%%%%%%%%%%%%%%%%%%%%%%%%%%%%%%%%%%%
% new commands
\newcommand{\nc}{\newcommand}
% operators
\renewcommand{\div}{\vec{\nabla}\! \cdot \!}
\newcommand{\grad}{\vec{\nabla}}
% latex shortcuts
\newcommand{\bea}{\begin{eqnarray}}
\newcommand{\eea}{\end{eqnarray}}
\newcommand{\be}{\begin{equation}}
\newcommand{\ee}{\end{equation}}
\newcommand{\bal}{\begin{align}}
\newcommand{\eali}{\end{align}}
\newcommand{\bi}{\begin{itemize}}
\newcommand{\ei}{\end{itemize}}
\newcommand{\ben}{\begin{enumerate}}
\newcommand{\een}{\end{enumerate}}
% DGFEM commands
\newcommand{\jmp}[1]{[\![#1]\!]}                     % jump
\newcommand{\mvl}[1]{\{\!\!\{#1\}\!\!\}}             % mean value
\newcommand{\keff}{\ensuremath{k_{\textit{eff}}}\xspace}
% shortcut for domain notation
\newcommand{\D}{\mathcal{D}}
% vector shortcuts
\newcommand{\vo}{\vec{\Omega}}
\newcommand{\vr}{\vec{r}}
\newcommand{\vn}{\vec{n}}
\newcommand{\vnk}{\vec{\mathbf{n}}}
\newcommand{\vj}{\vec{J}}
% extra space
\newcommand{\qq}{\quad\quad}
% common reference commands
\newcommand{\eqt}[1]{Eq.~(\ref{#1})}                     % equation
\newcommand{\fig}[1]{Fig.~\ref{#1}}                      % figure
\newcommand{\tbl}[1]{Table~\ref{#1}}                     % table

\newcommand{\ud}{\,\mathrm{d}}
%%%%%%%%%%%%%%%%%%%%%%%%%%%%%%%%%%%%%%%%%%%%%%%%%%%%%%%%%%%%%%%%%%%%

\begin{document}

\begin{center}
{ \Large Answers to Reviewer \#1}
\end{center}

\bigskip

\noindent Ms. Ref. No. JCOMP-D-13-01486\\
Title: Discontinuous Diffusion Synthetic Acceleration for Sn Transport on 2D Arbitrary Polygonal Meshes \\
{\it Journal of Computational Physics},\\

\bigskip
\bigskip

{
\color{blue}
 Combination of MIP, PWLG and AMG is interesting and could have good applications for radiation transport. The contents are sufficiently novel and interesting to warrant publication. The manuscript is well-written. The authors gave appropriate references to the previous works.
}

Thank you.

\bigskip

{
\color{blue}
However, I want the authors address the following issues before a publication:\\

1. The statement that 'polygonal grids allow for a reduced numbers of unknowns' is not complete. It is meaningful only in the condition of not sacrificing the accuracy. The authors need to provide numerical supports or give up the statement completely. The first part in the third paragraph of Introduction is superficial in this sense.
}

We agree. We had this ``information'' first-hand from Marv Adams and asked him about a Reference to back-up this statement. It turns out we have no numerical results to support this conjecture so far. The testbed code we developed for this paper does not implement LD on triangles nor can handle a mix of LD and BLD discretized cells, both of which would be required to compare against polygonal grids solved with PWLD. Thus, we have followed your suggestion to give up the statement completely but will keep in mind that this study needs to be carried out. 

\bigskip


{
\color{blue}
2. The authors indeed provided several different grid types but few important types are missing in my opinion:
a) a polygon with number of vertices greater than 4 with bad aspect ratios (Section 5.2.4 is only quadrilateral!). The assumption that 'the polygonal cells do not deviate significantly from regular polygons' on Line 157 weakens the conclusion 'effective for cells with high-aspect ratios'. It seems to me the same h perpendicular for all sides for a distorted polygon is not good;
b) a degenerated polygons with more than one hanging node from AMR.
The authors imply the solver works with arbitrary shapes of polygons, but I think there ought to be constraints on the grid. For example, any polygonal cell in the grid has to be convex.
}


We added a test case with large aspect ratio cell for a mesh consisting of degenerated polygons (rectangle-shaped cells with up to 23 vertices treated as a ``23-gon'' --a polygon with 23 sides). That grid also has many hanging nodes.

We were limited to convex cells because our transport sweep algorithm did not allow for cycles in the graph of the sweep. We added Reference 32 where PWLD was employed to carry out Sn transport sweeps for non-convex polygonal cells. 

\bigskip


{
\color{blue}
3. The authors need to carefully examine the control parameters they used for AMG, like the upper bound of the two-grid condition number (line 228), and etc., so that the readers can have a better understanding on the performances of different AMG packages.
}

The AMG algorithms (Trilinos ML and AGMG) were used with their default options. Unless stated otherwise, ML is
used with all the parameters set to the values corresponding to the ``SA'' (smoothed aggregation) options. The ML manual recommends ``SA'' for symmetric positive definite systems. AGMG is used with default parameters. 
%The AGMG manual
%To change the parameters, you need to change them in the code and recompile
%everything... The default parameters are always recommended except if you are an
%''expert''.


\bigskip


{
\color{blue}
4. Missing results of CPU-times and memory usages especially memory consumed by the AMG packages.
}


We have added CPU times for each test cases. The times are split into: Preconditioner setup (if any), MIP-DSA solve, and Total time. 
We have added memory usage for the AGMG package (this information was not available in Trilinos).
%\begin{table}[!htbp]
%  \begin{center}
%    \caption{Comparison of different preconditioners for the homogeneous test
%    problem (polygonal cells) using SI as solver.}
%    \begin{tabular}{|c|c|c|c|c|c|c|}
%      \hline
%      & No-DSA & CG & PCG-SGS & PCG-ML/U & PCG-ML/MIS & AGMG \\
%      \hline
%      SI iterations   & 7311    & 23      & 23      & 23      & 23      & 23 \\
%      CG iterations   & NA      & 46262   & 16712   & 652     & 603     & 555 \\
%\hline
%Precond init (s)     & NA      & NA      & 0.06388 & 1.73379 & 8.0426  & 0.388   \\
%MIP calculation (s)  & NA      & 877.861 & 1263.31 & 198.63  & 191.989 & 31.242  \\
%Total calculation (s)& 42666.7 & 1060.53 & 1447.53 & 382.275 & 384.422 & 216.946 \\
%%Precond init (\%)               &  NA    &      NA &  0.00014972 &   0.0040636  &    0.01885 &  0.00090937 \\ 
%%MIP calculation (\%)            &  NA    &  2.0575 &      2.9609 &     0.46554  &    0.44997 &    0.073223 \\
%%Total calculation (\%)          &100     &  2.4856 &      3.3926 &     0.89596  &    0.90099 &     0.50847 \\
%\hline
%Precond init (\%)               &  NA    &      NA &   0.00015 &   0.0041  &    0.019&    0.00091 \\ 
%MIP calculation (\%)            &  NA    &  2.08   &      2.96 &     0.47  &    0.45 &    0.07    \\
%Total calculation (\%)          & 100    &  2.49   &      3.39 &     0.90  &    0.90 &    0.51    \\
%      \hline
%    \end{tabular}
%    \label{comparison_homog_poly_si}
%  \end{center}
%\end{table}

\bigskip


{
\color{blue}
\noindent
Some minor problems or typos are listed:

1. GMRes equations are given and mentioned in Introduction but numerical results are not presented;

2. MIP appeared in Abstract is not defined;

3. Line 67 an incomplete sentence;

4. Add 'with spatial discretization' after 'Equation (1)' above Eq. (3);

5. Line 81, L is the streaming plus collision operator;

6. Check the grammar on Line 177;

7. Line 192-194, what is the different between n gamma and gamma?

8. Line 205-206, what is the subscript k?

9. To my knowledge, Gauss-Legendre-Chebyshev is a production quadrature, then why is there only one parameter Sn?

10. In the results section, what quantity those iteration tolerances are applied on?

11. Line 295, remove extra 'of';

12. Line 298, typo PGC should be PCG;

13. Line 340-341, what is the 'additional elementary matrices'?

14. Line 343, remove extra 'for'.}



\begin{enumerate}
\item Added a table for GMRes results with polygonal cells.
\item Corrected.
\item Corrected.
\item Added.
\item Corrected.
\item Corrected.
\item They both should be $n_\gamma$, corrected.
\item Corrected.
\item It is the triangular GLC quadrature. We have added reference to it (Ref. 30).
\item Corrected.
\item Corrected.
\item Corrected.
\item Added a paragraph at the end of Section 3 so that the statement in the conclusion is clearer.
\item We did not find an extra 'for'.
\end{enumerate}

\noindent 
Thank you for pointing these out.

\end{document}

