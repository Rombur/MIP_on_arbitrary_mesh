\documentstyle[11pt]{letter}

%%%%%% Letter Size Setup %%%%%%%%%%%%%%%%%%%%%%%%%%%%%%%%%%%%%%%%%%%%%%%%%%
%        \addtolength{\textwidth}{2.5cm}     %% For longer or shorter text width
        \addtolength{\topmargin}{-2.5cm}    %% For more or less top margin
       \addtolength{\textheight}{5cm}    %% For longer or shorter textheight
%        \addtolength{\oddsidemargin}{-1.25cm} %% For odd side margin (twoside)
                                            %% or margin (oneside)

\address{Dr. Jean Ragusa \\
Nuclear Engineering Dept.\\
Texas A\&M University\\
College Station, TX
 \vspace{0.5cm}}

%%%%%% The Signature  and Date %%%%%%%%%%%%%%%%%%%%%%%%%%%%%%%%%%%%%%%%%%%%

\signature{Bruno Turcksin and Jean Ragusa}


\begin{document}

\begin{letter}{Professor William Martin, \  Editor,\\
    Journal of Computational Physics,\\
\textbf{Re: Ref. No. JCOMP-D-13-01486}}


\date{\today}
%%%%%% More vertical space can be added here %%%%%%%%%%%%%%%%%%%%%%%%%%%%%%
%         \vspace{3.0cm}

\opening{Dear Dr. Martin,}
         \vspace{0.25cm}
%%%%%% More vertical space can be added here %%%%%%%%%%%%%%%%%%%%%%%%%%%%%%

Attached please find a revised version of our manuscript entitled
``Discontinuous Diffusion Synthetic Acceleration for Sn Transport on 2D Arbitrary Polygonal Meshes'' 
by B. Turcksin and J. C. Ragusa, for
publication in the Journal of Computational Physics. We appreciate the comments of the reviewers as we
believe they have allowed us to improve the paper.

We have attached a file of detailed responses for each reviewer.
We believe that we have adequately addressed all of their concerns and hope that our paper will be found suitable for publication.

%In our revision, we have addressed the issues raised by the
%reviewers per their suggestions. 
%
%% Removed per suggestion of JEM, assuming we include separate files for each reviewer
%
%%%Below is a summary of our responses to their suggestions.
%%%
%%%%%%%%%%%%%%%%%%%%%%%%%%%%%%%%%%%%%%%%%%%%%%%%%%%%%%%%%%%%%%%%%%%%%%%%%%%%%%%%%
%%%
%%%{\large \bf Review \#1}
%%%
%%%We have added a fixed source test problem to address this reviewer's largest comment.
%%%This suggestion allowed for a deeper understanding of how evaluate terms that our manuscript did not initially consider.
%%%All other comments were minor and were incorporated into the manuscript.
%%%
%%%%%%%%%%%%%%%%%%%%%%%%%%%%%%%%%%%%%%%%%%%%%%%%%%%%%%%%%%%%%%%%%%%%%%%%%%%%%%%%%
%%%
%%%{\large \bf Review \#2}
%%%
%%%The reviewer's comments have allowed us to create a manuscript that is clearer and properly attributes the historical origins of ideas that we expand upon in our paper.  Most importantly, the reviewer pointed out an error in Fig. 1 which we were able to correct.
%%%
%%%Reviewer \#2 raised two major issues that needed to be addressed prior to being suitable for publication.  
%%%
%%%First, Reviewer \#1 noted that Fig. 1  appeared to show that all of our numerical schemes were not convergent.
%%%We initially thought that the reviewer failed to note that the outflow plots in Figure 1 have a minimum of $h=1$, when all methods are convergent to the analytical solution only as $h\to 0$.
%%%However, in verifying this, we discovered an error in the script used to generate the data for Fig. 1.  We have since modified the script to collect data in the same way data was collected for all of our computational results data.
%%%
%%%Finally, we briefly respond to Reviewer \#2's comments regarding the Nuclear Science and Engineering audience's interest in our work.
%%%\begin{enumerate}
%%%\item ``authors make repeated reference to the benefits of mass lumping to radiative transfer problems'' Radiative transfer is mentioned in the introduction as a field that uses mass matrix lumping, in order to motivate why mass matrix lumping should be a research topic.  The only other mention of radiative transfer occurs in the conclusion's future work section.
%%%\item ``accuracy results, including error order, are not applicable to multidimensional configurations where the convergence order is limited by solution irregularity'' Yes, this is true, but multidimensional accuracy is also mesh dependent.  If multidimensional accuracy results are not universal, should they also not be published?
%%%\item ``robustness results are not extendable to multidimensional configurations'' In the final paragraph of the conclusion, we make this statement ourselves.  However, we point out that our methods provide a framework for generalized matrix lumping that is currently used in multi-dimensional calculations.
%%%\end{enumerate}

%%%%%%%%%%%%%%%%%%%%%%%%%%%%%%%%%%%%%%%%%%%%%%%%%%%%%%%%%%%%%%%%%%%%%%%%%%%%%%
\bigskip

We remain available for any further questions, should there be any.
\vspace{0.25cm}


%%%%%% More vertical space can be added here %%%%%%%%%%%%%%%%%%%%%%%%%%%%%%

%%%%%%% The Closing %%%%%%%%%%%%%%%%%%%%%%%%%%%%%%%%%%%%%%%%%%%%%%%%%%%%%%%
\closing{Best regards, }

\end{letter}

\end{document}

