%%%%%%%%%%%%%%%%%%%%%%%%%%%%%%%%%%%%%%%%%%%%%%%%%%%%%%%%%%%%%%%%%%%%%%%%%%%%%%%%%%
%%%%%%%%%%%%%%%%%%%%%%%%%%%%%%%%%%%%%%%%%%%%%%%%%%%%%%%%%%%%%%%%%%%%%%%%%%%%%%%%%%
\section{Discretization and Solution Techniques for the $S_n$ Transport Equation}\label{sec_transport}
%%%%%%%%%%%%%%%%%%%%%%%%%%%%%%%%%%%%%%%%%%%%%%%%%%%%%%%%%%%%%%%%%%%%%%%%%%%%%%%%%%
%%%%%%%%%%%%%%%%%%%%%%%%%%%%%%%%%%%%%%%%%%%%%%%%%%%%%%%%%%%%%%%%%%%%%%%%%%%%%%%%%%

We recall the \sn transport equations, the iterative techniques
employed to solve them, and discuss the PWLD discontinuous spatial
discretization for the transport equation with an emphasis on arbitrary
polygonal/polyhedral grids.

%%%%%%%%%%%%%%%%%%%%%%%%%%%%%%%%%%%%%%%%%%%%%%%%%%%%%%%%%%%%%%%%%%%%%%%%%%%%%%%%%%
\subsection{The \sn Transport Equations}
%%%%%%%%%%%%%%%%%%%%%%%%%%%%%%%%%%%%%%%%%%%%%%%%%%%%%%%%%%%%%%%%%%%%%%%%%%%%%%%%%%
Given an angular quadrature set $\{\bo_m,w_m\}_{1\leq m \leq M}$, the one-group
\sn transport equation with isotropic source and scattering is:
\begin{equation}
  \( \bo_m\cdot \bn + \Sigma_t(\br) \) \psi_m (\br) = \frac{1}{4\pi} \left( \Sigma_s
  (\br) \phi(\br) + S (\br) \right), \ \textrm{ for } \br \in \mc{D},\
  1 \leq m \leq M,
  \label{transport_sn}
\end{equation}
with $\psi_m(\br) = \psi(\br,\bo_m)$ the angular flux at position $\br$ in
direction $\bo_m$, $\Sigma_t$ and $\Sigma_s$ the total and scattering cross
sections, respectively, and $\mc{D}$ the spatial domain. The scalar flux is
defined and evaluated as follows:
\begin{equation}
  \phi(\br) \equiv \int_{4\pi} \psi(\br,\bo) d\bo \approx \sum_{m=1}^M w_m
  \psi_m (\br).
\end{equation}
For brevity, we assume only incoming boundary conditions, that is, $\psi_m(\br_b) =
\psi_m^{inc}(\br_b)$ for any point on the inflow boundary: $\br_b \in \partial \mc{D}_m^-
= \{\partial \mc{D} \textrm{ such that }\bo_m \cdot \bs{n}_b <0\}$, with
$\bs{n}_b = \bs{n}(\br_b)$ the outward unit normal vector at $\br_b$. 
\Cref{transport_sn} can be written in a compact form using operators:
\begin{align}
  & \bs{L} \Psi = \bs{M \Sigma}\Phi + S \equiv q, \label{L_Psi}\\
  &\Phi = \bs{D} \Psi, \label{Phi}
\end{align}
where $\Psi$ is the vector of angular fluxes, $\Phi$ the scalar flux,
$q$ is the total (scattering+external) source, $\bs{L}$ is the streaming
operator, $\bs{\Sigma}$ is the scattering matrix, $\bs{M}$ is the
moment-to-direction operator, and $\bs{D}$ is the direction-to-moment
operator. $\bs{L} = diag(\bs{L}_1,\hdots,\bs{L}_m,\hdots,\bs{L}_M)$ is 
a block-diagonal operator; given a total source $q$, one can solve independently 
for the resulting angular
fluxes in all directions. The action of $\bs{L}^{-1}$ is often referred to as
a transport sweep when discontinuous spatial approximations are
employed because, for any direction $\bo_m$, the action of $\bs{L}_m^{-1}$ can
be obtained by traversing the mesh (i.e., sweeping) in a specific ordering of
the cells, thus requiring only to solve a small linear system of equations,
cell by cell. The order in which the elements are solved constitutes the graph
of the sweep; for brevity and since this is not the focus of this article,
we do not expand on situations where the graph presents dependencies
(cycles); in such a case, these dependencies can either be lagged within the
iterative procedure \cite{dgfem} or the solution vector consisting of the scalar 
flux is augmented by the angular flux unknowns that cause the cycle \cite{mip}.

%%%%%%%%%%%%%%%%%%%%%%%%%%%%%%%%%%%%%%%%%%%%%%%%%%%%%%%%%%%%%%%%%%%%%%%%%%%%%%%%%%
\subsection{Solution Techniques}
%%%%%%%%%%%%%%%%%%%%%%%%%%%%%%%%%%%%%%%%%%%%%%%%%%%%%%%%%%%%%%%%%%%%%%%%%%%%%%%%%%

\Cref{L_Psi,Phi} can be solved using the Source Iteration (SI) method (a
stationary iterative technique also known as Richardson iteration). 
The $\ell^{th}$ iteration of the SI technique is given by :
\begin{equation}
  \Phi^{(\ell+1)} = \bs{DL}^{-1} \(\bs{M\Sigma}\Phi^{(\ell)} + S\).
\end{equation}
Alternatively, a subspace Krylov method (usually GMRes) can be employed to
solve the linear system of equations, recast of follows:
\begin{equation}
  \(\bs{I} - \bs{DL}^{-1}\bs{M \Sigma}\) \Phi = \bs{DL}^{-1}S
\end{equation}
SI and GMRes-based transport solves require transport sweeps (the action of $\bs{L}^{-1}$).
When the scattering ratio
$c=\frac{\Sigma_s}{\Sigma_t}$ tends to one in optically thick domains, the
number of SI and GMRes iterations can become large. To speed up convergence, a DSA
preconditioner is needed; this is further discussed in \Cref{sec_mip}.

%%%%%%%%%%%%%%%%%%%%%%%%%%%%%%%%%%%%%%%%%%%%%%%%%%%%%%%%%%%%%%%%%%%%%%%%%%%%%%%%%%
\subsection{Discontinuous Finite Element Discretization on Arbitrary Grids}
%%%%%%%%%%%%%%%%%%%%%%%%%%%%%%%%%%%%%%%%%%%%%%%%%%%%%%%%%%%%%%%%%%%%%%%%%%%%%%%%%%

To seek a DFE solution for the angular flux, the domain $\mc{D}$ is partitioned 
into arbitrary polygonal cells. For 
a given streaming direction $\bo_m$, the discontinuous finite
element formulation, on a given spatial cell $K$, is given by:
\begin{equation}
  -\int_{K} \(\psi_m \bo_m \cdot \bn \tf + \Sigma_t \Psi_m \tf \)\ d\br +
  \int_{\partial K^+} \bo_m \cdot \bs{n} \Psi_m \tf \ d\br = \int_{K} q\, \tf \ d\br +
  \int_{\partial K^{-}} |\bo_m \cdot \bs{n}| \psi_m^{\uparrow}\, \tf \ d\br,
  \label{transport_int}
\end{equation}
where $\tf$ represents a generic discontinuous finite element basis function, $\partial K^{-}$ 
is the inflow face of element $K$, $\partial K^{+}$ is the outflow face of 
element $K$. The angular flux values on an inflow face, denoted by 
$\psi_m^{\uparrow}$ in \cref{transport_int}, are taken from the upwind neighbor 
element of that face.

Next, we define the $\tf$ basis functions for the PWLD discretization. First, we
define the within-cell point $c$ for any 2D polygonal cell: the coordinates 
of $c$ are the weighted averages of the polygon's vertices:
\begin{equation}
  u_c = \sum_{i=1}^{N_V} \alpha_i u_i,
\end{equation}
where $u=x$ or $y$, $N_V$ is the number of vertices for the cell under
consideration, and the (positive) weights $\alpha_i$ are such that $\sum_{i=1}^{N_V} \alpha_i =1$. 
The basis function at vertex $i$ is defined by 
\cite{pwld_2d}:
\begin{equation}
  b_i(x,y) = t_i(x,y) + \alpha_i t_c(x,y),
\end{equation}
where $t_i(x,y)$ is a linear function such that $t_i(x,y)$ is unity at vertex
$i$ and zero at vertices $i-1$, $i+1$, and $c$. The $t_c(x,y)$ function is a ``tent''
function in the interior of the cell; it is unity at the within-cell point $c$
and zero at all of the vertices of the cell. The number of PWLD basis functions is 
equal to the number of vertices in the polygon (i.e., $1 \le i \le N_V$). In this paper, the arbitrary
positive weights $\alpha_i$ are chosen to be equal to $\frac{1}{N_V}$; for example, on a
quadrilateral cell, we employ $\alpha_i =\frac{1}{4}$. Finally, note that on
triangular cells, the PWLD basis functions reduce to the
standard Linear Discontinuous (LD) basis functions if ones chooses $\alpha_i = \frac{1}{3}$. 
Given the definition of the PWLD
finite elements, it may seem complicated to build the elementary transport 
matrices on an arbitrary polygonal cell but the construction of such matrices
can be greatly simplified using the notion of ``sub-cells'', where a ``sub-cell'',
in 2D, is defined as the triangular cell linking an edge of the polygonal cell to its
within-cell point. Note that these ``sub-cells'' are never created as part of the 
polygonal grid; they are just \textcolor{red}{a simple implementation means of computing}
the elementary matrices of a polygonal cell by looping over its sub-cells. 
