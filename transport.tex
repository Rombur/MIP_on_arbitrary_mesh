\section{Discretization and Solution Techniques for the $S_n$ Transport
Equation}\label{sec_transport}
Here, we briefly recall the \sn transport equations, the iterative techniques
employed to solve them, and discuss the PWLD discontinuous spatial
discretization for the transport equation with an emphasis on arbitrary
polygonal/polyhedral grids.
\subsection{The \sn Transport Equations}
Given an angular quadrature set $\{\bo_d,w_d\}_{1\leq d\leq M}$, the one-group
\sn transport equation with isotropic source and scattering is:
\begin{equation}
  \(\bo_d\cdot \bn + \Sigma_t(\br)\)\psi_d (\br) = \frac{1}{4\pi} \Sigma_s
  (\br) \phi(\br) + \frac{1}{4\pi} S (\br), \ \textrm{ for } \br \in \mc{D},\
  1 \leq d \leq M
  \label{transport_sn}
\end{equation}
with $\psi_d(\br) = \psi(\br,\bo_d)$ the angular flux at position $\br$ in
direction $\bo_d$, $\Sigma_t$ and $\Sigma_s$ the total and scattering cross
sections, respectively, and $\mc{D}$ the spatial domain. The scalar flux is
defined as and evaluated as follows:
\begin{equation}
  \phi(\br) \equiv \int_{4\pi} \psi(\br,\bo) d\bo \approx \sum_{d=1}^M w_d
  \psi_d (\br).
\end{equation}
For brevity, we assume only incoming boundary conditions, $\psi_d(\br_b) =
\psi_d^{inc}(\br_b)$, for any boundary location $\br_b \in \partial \mc{D}_d^-
= \{\partial \mc{D} \textrm{ such that }\bo_d \cdot \bs{n}_b <0\}$, where
$\bs{n}_b = \bs{n}(\br_b)$ is the outward unit normal vector at the boundary. 

\Cref{transport_sn} can be written in a compact form using operators:
\begin{align}
  & \bs{L} \Psi = \bs{M \Sigma}\Phi + S \equiv q, \label{L_Psi}\\
  &\Phi = \bs{D} \Psi, \label{Phi}
\end{align}
where $\Psi$ is the vector of angular fluxes, $\Phi$ the vector scalar flux,
$q$ is the total (scattering+external) source $\bs{L}$ is the streaming
operator, $\bs{M}$ is the moment-to-direction operator, and $\bs{D}$ is the
direction-to-moment operator. $\bs{L} = diag
(\bs{L}_1,\hdots,\bs{L}_d,\hdots,\bs{L}_M)$ is a diagonal operator; given a
total cross section, one can solve independently for the resulting angular
fluxes in all directions. The action of $\bs{L}^{-1}$ is often referred to as
a \emph{transport sweep} when discontinuous spatial approximations are
employed because, for any direction $\bo_d$, the action of $\bs{L}_d^{-1}$ can
be obtained by traversing the mesh (i.e., sweeping) in a specific ordering of
the cells, thus one needs only to solve a small linear system of equations,
cell by cell. The order in which the elements are solved constitutes the graph
of the sweep; for brevity and because this is not the focus of this article,
we do not expand on situations where the graph can present some dependencies
(cycles); in such a case, these dependencies can either be lagged within the
iterative procedure or the solution vector consisting of the scalar flux is
augmented by the angular fluxes that cause the cycle.
\subsection{Solution Techniques}
\Cref{L_Psi,Phi} can be solved using the Source Iteration (SI) method (a
stationary iterative technique also knows as Richardson iteration). The SI
techniques at the $\ell^{th}$ iteration is given by :
\begin{equation}
  \Phi^{(\ell+1)} = \bs{DL}^{-1} \(\bs{M\Sigma}\Phi^{(\ell)} + S\)
\end{equation}
Alternatively, a subspace Krylov method (usually GMRes) can be employed to
solve the system of equations:
\begin{equation}
  \(\bs{I} - \bs{DL}^{-1}\bs{M \Sigma}\) \Phi = \bs{DL}^{-1}S
\end{equation}
Both the SI and the GMRES approaches require transport sweeps (the action of
$\bs{L}^{-1}$ is required in both procedures).

When the scattering ratio $c=\frac{\Sigma_s}{\Sigma_t}$ tends to one in
optically thick domains, the number of SI and GMRES can become large. To
accelerate the convergence, a DSA preconditioner is needed; the MIP
discontinuous finite element discretization of the diffusion equation for
arbitrary polygonal/polyhedral grids is presented in \Cref{sec_mip}.
\subsection{Discontinuous Finite Element Discretization on Arbitrary Grids}
The domain $\mc{D}$ is meshed into element $K$ (specifically polygons and
polyhedra). For a given streaming direction $\bo_d$, the discontinuous finite
element scheme on a given element $K$ is given by:
\begin{equation}
  -\int_{K} \(\psi_d \bo_d \cdot \bn b + \Sigma_t \Psi_d b \)\ d\br +
  \int_{\partial K^+} \bo_d \cdot \bs{n} \Psi_d b\ d\br = \int_{K} qb\ d\br +
  \int_{\partial K^{-}} |\bo_d \cdot \bs{n}| \psi_d^{\uparrow}b \ d\br
  \label{transport_int}
\end{equation}
where $b$ represents a generic basis function, $\partial K^{-}$ is the inflow
face of element $K$, $\partial K^{+}$ is the outflow face of element $K$. The
angular flux values on an inflow face, denoted by $\psi_d^{\uparrow}$ in
\cref{transport_int}, are taken from the upwind neighbor element of that face.

Next, we define the basis function $b$ for the PWLD method. First, we
introduce a within-cell point $c$ inside the polygon in 2D (resp., polyhedron
in 3D). The coordinates of $c$ are weighted averages of the vertex
coordinates:
\begin{equation}
  u_c = \sum_{j=1}^{N_V} \alpha_j u_j
\end{equation}
where $u=x$, $y,$ or $z$, $N_V$ is the number of vertices for the cell under
consideration, and the weights are such that $\sum_{j=1}^V \alpha_j =1$,
$\alpha_j \geq 0\ \forall j$. In 2D, the basis function at vertex $j$ is
defined by \cite{pwld_2d}:
\begin{equation}
  b_j(x,y) = t_j(x,y) + \alpha_j t_c(x,y)
\end{equation}
where $t_j(x,y)$ is a linear function such that $t_j(x,y)$ is unity at vertex
$j$ and zero at $j-1$, $j+1$, and $c$. The $t_c(x,y)$ function is a ``hat''
function in the interior of the cell, is unity at point $c$, and zero at all
the vertices of the cell. In 3D, the same definition applies for $t_c$ but
$t_j$ is now constructed on a tetrahedron obtained from two adjacent vertices,
the cell center point $c$, and a face center point, where that face contains
both vertices. The PWLD method employs a notion of sub-cells on each arbitrary 
polygon (polyhedron); the number of PWLD basis functions is equal to the
number of vertices in the polygon (polyhedron). In this paper, the arbitrary
positive weights $\alpha_j$ are chosen to be $\frac{1}{N_V}$, thus, on a
quadrilateral cell, one has $\alpha_j =0.25 \forall j$. Finally, note that on
triangular (resp., tetrahedral) cells, the PWLD basis functions reduce to the
standard Linear Discontinuous (LD) basis functions of $\alpha_j = \frac{1}{3}$
$\(\textrm{resp., }\alpha_j = \frac{1}{4}\)$. Given the definition of the PWLD
finite elements, it may seem complicated to build the mass and the gradient
matrices on an arbitrary polygonal cell but the construction of such matrices
can be greatly simplified using the sub-cells. By looping over all of the
sub-cells of a cell, the elementary matrices of the cell can easily be
computed.
