\appendix
\section{Building the mass matrix}
Let us define the mass matrix $M_{PWLD}$ and the mass matrix on a
``side'' sub-cell $M$. We assume that the third basis function is associated
to the middle point of the cell. The first (respectively second) basis function
on the cell correspond to the first (respectively second) basis function of
the ``side'' sub-cell.

{\allowdisplaybreaks
\begin{align}
& M_{PWLD}(0,0) = M_{PWLD}(0,0) + M(0,0) + \alpha M(0,2) + \alpha M(2,0) + 
\alpha^2M(2,2)\\
& M_{PWLD}(0,1) = M_{PWLD}(0,1) + M(0,1) + \alpha M(0,2) + \alpha M(2,1) + 
\alpha^2M(2,2)\\
& M_{PWLD}(0,2) = M_{PWLD}(0,2) + \alpha M(0,2) + \alpha^2 M(2,2)\\
& M_{PWLD}(0,3) = M_{PWLD}(0,3) + \alpha M(0,2) + \alpha^2M(2,2)\\
& M_{PWLD}(1,0) = M_{PWLD}(1,0) + M(1,0) + \alpha M(1,2) + \alpha M(2,0) + 
\alpha^2M(2,2)\\
& M_{PWLD}(1,1) = M_{PWLD}(1,1) + M(1,1) + \alpha M(1,2) + \alpha M(2,1) + 
\alpha^2M(2,2)\\
& M_{PWLD}(1,2) = M_{PWLD}(1,2) + \alpha M(1,2) + \alpha^2M(2,2)\\
& M_{PWLD}(1,3) = M_{PWLD}(1,3) + \alpha M(1,2) + \alpha^2M(2,2)\\
& M_{PWLD}(2,0) = M_{PWLD}(2,0) + \alpha M(2,0) + \alpha^2M(2,2)\\
& M_{PWLD}(2,1) = M_{PWLD}(2,1) + \alpha M(2,1) + \alpha^2M(2,2)\\
& M_{PWLD}(2,2) = M_{PWLD}(2,2) + \alpha^2M(2,2)\\
& M_{PWLD}(2,3) = M_{PWLD}(2,3) + \alpha^2M(2,2)\\
& M_{PWLD}(3,0) = M_{PWLD}(3,0) + \alpha M(2,0) + \alpha^2M(2,2)\\
& M_{PWLD}(3,1) = M_{PWLD}(3,1) + \alpha M(2,1) + \alpha^2M(2,2)\\
& M_{PWLD}(3,2) = M_{PWLD}(3,2) + \alpha^2M(2,2)\\
& M_{PWLD}(3,3) = M_{PWLD}(3,3) + \alpha^2M(2,2)
\end{align}}    
The mass matrix $M$ is given by:
\begin{equation}
M = \frac{A}{12}
\begin{pmatrix}
2 & 1 & 1\\
1 & 2 & 1\\
1 & 1 & 2
\end{pmatrix}
\end{equation}
where $A$ is the area of the ``side'' sub-cell.
\section{ML options}
The options used to solve MIP with ML are \cite{ml_guide}:
\begin{description}
\item [Uncoupled:] Attempt to construct aggregates of optimal size ($3^d$
nodes in $d$ dimensions). Each process works independently and aggregates
cannot span processes.
\item [MIS:] Uses maximal independent set techniques \cite{mis} to define
aggregates. Aggregates can span processes. May provide better quality
aggregates than \bf{Uncoupled}, but computationally more expensive because it
requires matrix-matrix product.
\item[Symmetric Gauss-Seidel]
\item[Amesos-KLU:] Use \bf{KLU} through \bf{Amesos}. Coarse grid problem is
shipped to proc 0, solved and solution is broadcast.
\end{description}
The MultiLevelPreconditioner class provides default values for five different
preconditioner types, including classical smoothed aggregation for symmetric
positive definite or nearly symmetric definite systems.
\begin{description}
\item [option name:] SA
\item [max levels:] 10
\item [prec type:] V-cycle
\item [aggregation type:] Uncoupled-MIS
\item [aggregation dumping factor:] 4/3
\item [eigen-analysis type:] cg
\item [eigen-analysis iterations:] 10
\item [smoother sweeps:] 2
\item [smoother damping factor:] 1.0
\item [smoother pre or post:] both
\item [smoother type:] symmetric Gauss-Seidel
\item [coarse type:] Amesos-KLU
\item [coarse max size:] 128
\end{description}
