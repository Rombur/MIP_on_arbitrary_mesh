\section{Introduction}

In this article, we present a Diffusion Synthetic Acceleration (DSA) scheme
that is fully compatible with the piecewise linear discontinuous finite
element (PWLD) discretization of the transport equation on arbitrary
polygonal cells. Before discussing the PWLD discretization applied
to the transport equation on such arbitrary grids, we first recall the
rationale for acceleration schemes (i.e., preconditioners) when solving
iteratively transport problems.

%%%%%%%%%%%%%%%%%%%%%%%%%%%%%%%%%%%%%%%%%%%%%%%%%%%%%%%%%%%%%%%%%%%%%%%%%%%%%%%%%%
%\subsection{Rationale for DSA preconditioning}
%%%%%%%%%%%%%%%%%%%%%%%%%%%%%%%%%%%%%%%%%%%%%%%%%%%%%%%%%%%%%%%%%%%%%%%%%%%%%%%%%%

Because analytical and closed form solutions are unavailable for most
radiation transport problems of practical interest, one typically employs
iterative techniques to solve the large system of equations that results from
the spatial and angular discretizations of the transport equation. Standard
iterative techniques for the first-order form of the discrete-ordinate (\sn)
transport equation include the Source Iteration (SI) technique and Krylov 
subspace algorithms (usually GMRES \cite{gmres}). For highly diffusive materials 
(i.e., with scattering ratios $c=\Sigma_s / \Sigma_t $ close to 1) and optically 
thick configurations (i.e., not leakage dominated), these iterative techniques 
can become quite ineffective, requiring high iteration counts and possibly 
leading to false convergence. However, SI and GMRES-based transport solves 
can be effectively accelerated (preconditioned) with DSA approaches 
\cite{dsa_ref,larsen_dsa,consistent_p1,m4s,wla,mip}. 


It is well established that the spatial discretization of the DSA equations
must be somewhat ``consistent'' with the one used for the \sn transport equations to
yield unconditionally stable and efficient DSA schemes
\cite{dsa_ref,larsen_dsa,consistent_p1,m4s,wla,mip}. However, the search for full
consistency between the discretized transport equations and the discretized
diffusion may not be computationally practical (especially for unstructured
arbitrary meshes, \cite{dsa_ref}). For instance, Warsa, Wareing, and
Morel \cite{consistent_p1} derived a fully consistent DSA scheme for linear
discontinuous finite elements on unstructured tetrahedral meshes; their DSA
scheme yielded in a $P_1$ system of equations which was found to be
computationally more expensive than partially consistent DSA schemes that are
based upon discretizations of a standard diffusion equation. Some partially 
consistent schemes have been analyzed for discontinuous finite element
(DFE) discretizations of the transport equation on unstructured meshes, for
instance, the modified-four-step (M4S) scheme \cite{m4s}, the
Wareing-Larsen-Adams (WLA) scheme \cite{wla}, and the Modified Interior
Penalty (MIP) scheme \cite{mip}.

%%%%%%%%%%%%%%%%%%%%%%%%%%%%%%%%%%%%%%%%%%%%%%%%%%%%%%%%%%%%%%%%%%%%%%%%%%%%%%%%%%
%\subsection{PWLD discretization on arbitrary grids}
%%%%%%%%%%%%%%%%%%%%%%%%%%%%%%%%%%%%%%%%%%%%%%%%%%%%%%%%%%%%%%%%%%%%%%%%%%%%%%%%%%

Several discretization methods haven been developed for 
arbitrary polygonal meshes \cite{pwld_2d,pwld_3d,cfm_dfm,pwl_diffusion,
palmer_fe,mimetic,cell_centered_diff,palmer_proc,palmer_ane,wachspress,pwbld}.
In this work, we focus on the PWLD discretization because it was successfully
used to discretize the transport equation \cite{pwld_2d,pwld_3d}. This
discretization can be applied for any polygonal cells and the integrals
generated by this discretization can be easily computed analytically. 

As of today, a lot of the ongoing effort to develop a DSA scheme for 
polygonal cells has focused on adapting the WLA scheme to polygonal meshes
\cite{cfm_dfm,wla_pwl}. The WLA scheme is a two-stage process, where first a
diffusion solution is obtained using a {\em continuous} finite element
discretization and then a {\em discontinuous } update is performed cell-by-cell 
in order to provide an appropriate discontinuous scalar flux correction to the DFE transport 
solver. In \cite{consistent_p1}, the WLA scheme was
found to be a stable and effective DSA technique, though its efficiency
degraded as the problem became more optically thick and highly diffusive.
To the authors' best knowledge, no work is currently done to adapt the M4S 
technique to polygonal meshes. This is probably due to the fact
that even though the scheme is effective in one-dimensional slab and
two-dimensional rectangular geometries, it was found to be divergent for
three-dimensional tetrahedral meshes with linear discontinuous elements.
Furthermore, the scheme does not yield a Symmetric Positive Definite (SPD)
matrix. In this paper, we present an extension of the MIP technique to the
PWLD discretization technique for arbitrary polygonal meshes.
The MIP scheme is based on the standard Interior Penalty (IP) method for the
discontinuous discretization of diffusion equations. MIP was first derived in
\cite{mip}, where it was applied to triangular unstructured meshes (with
locally adapted cells). MIP did not suffer the same problems than WLA when the
problem becomes optically thick and highly diffusive and it is therefore an
interesting alternative to WLA. Because MIP produces SPD equations, it has been 
solved using a Preconditioned Conjugate Gradient (PCG) technique preconditioned 
using a symmetric successive over-relaxation method (SSOR) in \cite{mip}. Here, 
the effectiveness of algebraic multigrid methods (AMG) to precondition diffusion
solver \cite{amg,amg_course} will be tested and compared with PCG+SGS.
Algebraic multigrid methods allow the use of multigrid techniques when no grid
information is available or when the grid is unstructured. Instead of using a
succession of grids based on the geometry of the problems, the ``grid levels''
are based on properties of the matrix.

The remainder of this paper is organized as follows. In \Cref{sec_transport},
we briefly recall the \sn transport equation, its discontinuous finite element
discretization using the PWLD technique, and Source Iteration. In \Cref{sec_mip}, 
we recall the MIP scheme and adapt it to the PWLD discretization for arbitrary 
polygons/polyhedra. In \Cref{sec_amg}, we introduce the Algebraic MultiGrid (AMG) 
approaches used here: the ML package of Trilinos \cite{ml_guide} and the
AGMG (AGgregation-based algebraic MultiGrid) technique \cite{agmg_guide}. In
\Cref{sec_res}, we present a Fourier analysis for the MIP scheme discretized with
PWLD and we compare the different AMG with preconditioned CG solver.
Conclusions are given in \Cref{sec_conc}.
